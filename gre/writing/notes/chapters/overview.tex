\chapter{简介}

\section{组成部分}

  写作分两个部分:

  \begin{itemize}
    \item \textbf{\say{Issue}};
    \begin{itemize}
      \item 考察: 批判性思维逻辑, 洞察力;
    \end{itemize}

    \item \textbf{\say{Argument}}:
    \begin{itemize}
      \item 相对简单;
      \item 考察: 逻辑思维能力;
    \end{itemize}
  \end{itemize}

\section{字数}

  \begin{itemize}
    \item 写作每部分30分钟;
    \item 400-600字;
    \item 官方不限定字数;
  \end{itemize}

\section{出题}

  \begin{itemize}
    \item 相比托福作文难度更大;
    \item \textbf{写作题库}:
    \begin{itemize}
      \item 只考题库里的题;
      \item 没必要全写;
      \item 得分点为逻辑, 思维能力;
    \end{itemize}
  \end{itemize}

\section{技巧}

  \subsection{简称}

    \begin{itemize}
      \item 先写: \say{A (ABC)}
      \item 再引用 \say{A}
    \end{itemize}

\section{分数目标}

  \begin{itemize}
    \item 中国平均分: 3, 全球平均分: 3.5;
    \item \textbf{理工科研究生申请}:
    \begin{itemize}
      \item 最低 3.5;
      \item 最高: 4;
    \end{itemize}

    \item \textbf{文科研究生申请}:
    \begin{itemize}
      \item \textbf{最低}: 4;
      \item \textbf{最高}: 4.5;
    \end{itemize}
  \end{itemize}

\section{备考}

  \begin{enumerate}
    \item 分析题目: 至少分析1/4的题目
    \begin{itemize}
      \item Issue类每个类型分析3道
    \end{itemize}

    \item 读范文: 标注功能
    \item 练习: 两类题各写四篇
  \end{enumerate}

\section{资源}

  \begin{figure}[H]
    \centering
    \includegraphics[width=0.5\columnwidth]{images/保罗主义.png}
  \end{figure}
