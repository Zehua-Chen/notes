\documentclass{article}
\usepackage{newtxtext}
\usepackage{newtxmath}
\usepackage{listings}
\usepackage{soul}
\usepackage{amsmath}
\usepackage{dirtytalk}
\usepackage{bookmark}
\usepackage{tabu}
\setlength{\parindent}{0pt}
\setlength{\parskip}{6pt}
\newcommand{\important}[1]{{\color{MaterialOrange800} #1}}


\begin{document}
  \section{Description}

    Every individual in a society has a responsibility to obey just laws
    and disobey and resist unjust laws.

  \section{Outline}

    \subsection{Object Just Laws}

      \begin{itemize}
        \item \textbf{赞同}
        \begin{itemize}
          \item With laws that are just and fair being obeyed and defended,
          social stability is maintained
          \begin{itemize}
            \item Countries/regions where just is being trampled, anarchy
            prevails accordingly; Ex. Somalia, The Golden Triangle,
            Wartime Europe
          \end{itemize}

          \item Meanwhile, social progress becomes an easier prospect
          \begin{itemize}
            \item The US Constitution; The Bills of Rights
          \end{itemize}
        \end{itemize}

        \item \textbf{不赞同}
        \begin{itemize}
          \item \important{History has suggested that, collectively, lawmakers
          around the world would often distort the definition of justice}
          \begin{itemize}
            \item The final solution by Nazis
          \end{itemize}
        \end{itemize}
      \end{itemize}

    \subsection{Disobey unjust laws}

      \begin{itemize}
        \item \textbf{赞同}
        \begin{itemize}
          \item Fighting against laws that are in violation of ethical codes
          is an act of pursuing basic humans rights
          \begin{itemize}
            \item Slavery: Harriet Tubman - The Moses of Slaves
          \end{itemize}

          \item To promote our personal wellbeing ad that of the community we
          live in, both of which are taken by force of law from us, we have to
          challenge the injustice
          \begin{itemize}
            \item Stamp Act 1712 - stamp duty
            \item The Boston Tea Party - tea duty -> The American
            Revolutionary War
            \item Ghandi, salt
          \end{itemize}

          \item The pursuit of conscientious and peace of mind also
          important{compels one to choose disobedience or resistance}
          \begin{itemize}
            \item Schindler’s List
          \end{itemize}

          \item Disobeying unjust laws prevents tragedies (WWII)
        \end{itemize}

        \item \textbf{不赞同}
        \begin{itemize}
          \item A law may be deemed as unjust by an individual
          (i.e. subjective), but is in fact just considering from the
          perspective of the society
          \begin{itemize}
            \item Reservation of Native Americans
            \item Affordable Care Act
          \end{itemize}

          \item If everyone ie exercising their own senes of righteousness by
          disobeying and even resisting what in their mind is unjust or
          unethical \important{social chaos} more often than not ensues
          \begin{itemize}
            \item Greece: The Austerity Act
            \item South Koreans during Asian Financial Crisis 1998
          \end{itemize}

          \item Perhaps not every individual has the responsibility;
          \say{with great power comes great responsibility}
          \begin{itemize}
            \item Harriet Tubman vs Abraham Lincoln: 还是得靠大人物解决问题
          \end{itemize}

          \item Laws of the modern world evolved on their own through
          legal process of legislation
          \begin{itemize}
            \item Amendments to the United States Constitution
          \end{itemize}

          \item Under the rule of law, all members of society
          (including those in government) are considered equally subject to
          publicly disclosed legal codes and processes. Those who feel deprived
          of their deserved rights can simply file a lawsuit against the law
          they seem unjust
          \begin{itemize}
            \item Brown v. Board of Education
            \item New York Times Co. v. Sullvian
            \item Roe v. Wade
          \end{itemize}
        \end{itemize}
      \end{itemize}

  \section{Writeup}

    There are many countries in the world. All of them have some kind of laws,
    but some are in turmoil and poverty and others are in peace and prosperity.
    In most cases, those countries with peace and prosperity have a set of much
    well-enforced laws and those in turmoil and poverty do not. Therefore,
    laws should be followed regardless of if an individual think it is just
    or not.

    Just laws should always be followed, as following just laws can ensure
    prosperity. In the US, the constitution requires all those who seek
    to become public offices to go through the election process. Through this
    process, transfer of power in America has always been peaceful and
    the republic is upheld. If everyone
    that want to be presidents decide to launch a coup rather than try to
    convince people to vote for them, America would be in constant state of
    war and the union would be gone.

    The views of what is just and what is unjust vary greatly are subjective.
    What someone views as unjust could be for the betterment of the society.
    Almost everywhere, robbing is prohibited. However, when a drug addict needs
    more money to buy drugs, the law that protects private property became
    unjust. In this case, should the drug addict disobey what he considers
    to be unjust laws?

    However, not all unjust laws should be followed. When an unjust law
    violates the rights of others, then those laws must not be executed.
    Otherwise, tragedies would occur.
    When the Berlin Wall was built, it was ilegal for residents of East Berlin
    to cross those walls to seek a better life. For the East German soldiers
    guarding the wall, the law required them to shoot any traspassers.
    This in the end caused a great tragedy in human history. Should the
    guards disobey this unjust laws, this tragedy could have been prevented.

    In summary, laws should always be followed regardless if an individual
    perceive it as right or wrong, so long as the laws do not violate the
    rights of others.

\end{document}
