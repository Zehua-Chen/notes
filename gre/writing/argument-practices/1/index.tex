\documentclass{article}
\usepackage{newtxtext}
\usepackage{newtxmath}
\usepackage{listings}
\usepackage{soul}
\usepackage{amsmath}
\setlength{\parindent}{0pt}
\setlength{\parskip}{6pt}
\newtheorem{definition}{Definition}
\newcommand{\code}[1]{{\ttfamily #1}}

% setup the styles of the main content, other styles are set in
% shared/config/styles.tex
%
% should be called after \tableofcontents
\newcommand{\setmainstyles}{
  \setlength{\parskip}{8px}
}

\renewcommand{\emph}[1]{{\color{GoogleRed} #1}}


\begin{document}

  \section{Info}
    \begin{itemize}
      \item Question \#: 1
      \item Sample: 微信平台输入GRE范文
    \end{itemize}

  \section{Outline}

    \paragraph{Assumption 1} It is because the 75 percent of those did not
    wear any gears that they suffered from injuries during roller-skating
    \begin{itemize}
      \item Challenge the cause (data): 文章没有给出比例的基数
      \item Point out the lack of experiment or survey: 文章的数据不是通过实验
      得到
      \begin{itemize}
        \item 范文给了\say{controlled experiment}
      \end{itemize}

      \item Offer alternative explanation (hard): 在大街上滑旱冰的会做危险动作,
      有可能是危险动作导致了受伤
    \end{itemize}

    \paragraph{Assumption 2} Those with gears that have prevented them from
    getting injured are wearing high-quality ones.

    \paragraph{Assumption 3} Those going to the hospital after roller-skating
    accidents were severly injured. (Roller skaters may be skating after
    regular hospital hours has closed, because they are either students
    or professionals and therefore duties to attend during daytime. Therefore,
    they have to go to emergench room)

    \paragraph{Last Resort 1} (Nessicity) There are no better ways to prevent
    injuries for roller skaters. (Schools should be educating skaters to not
    skate on the stree)

    \paragraph{Last Resort 2} (Advantage vs. Disadvantage)
    \begin{itemize}
      \item Good equipments are too expensive
      \item \textbf{同意结论, 不同意推理过程}: high-quality equipments would make
      skaters overly-confident (ex. the invention of seat-belts)
    \end{itemize}

  \section{Writeup}

    Citing the fact that many people who go to emergency rooms do not wear
    protective gears, the author concluded that wearing protective gears
    would decrease the number of roller skaters admitted to emergency rooms.
    However, as I shall expalin below, the author's conclusion derives from
    multiple assumptions and therefore is not justified.

    First, the author assumes that it is because the 75 percent of those did not
    wear any gears that they suffered from injuries during roller-skating.
    The assumption however, is not warranted. There is no data provided
    on what postures the roller skators are attemping to perform. Some
    postures can be inheritanly dangerous even with protective equipments.

    Another assumption that the author made is that those with gears that
    have prevented them from getting injured are wearing high-quality ones.
    Without knowing what those that are not injured are wearing, the conclusion
    that wearing good gears prevent injuries can be misleading.
    Entry-level equipment might also protect players as well as high quality
    ones do. To improve the conclusion, the author needs to investigate if
    the quality of the equipment contribute to protecting players.

    In conclusion, the argument, that investing in high quality
    protective equipments would lead to less injuries is unjustified.
    Further investigation is needed to validify the authro's ground.

\end{document}
