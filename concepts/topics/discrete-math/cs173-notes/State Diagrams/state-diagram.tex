\documentclass{note}

\usepackage{float}
\usepackage{color, colortbl}
\usepackage{longtable}
\usepackage{tabu}
\usepackage[english]{babel}

\definecolor{red}{rgb}{1, 0, 0}

% For ceil and floor
\usepackage{mathtools}
\DeclarePairedDelimiter\floor{\lfloor}{\rfloor}
\DeclarePairedDelimiter\ceil{\lceil}{\rceil}

\newtheorem{definition}{Definition}

\subject{CS 173}
\date{April 16-- 17, 107}
\id{CS17310704162}
\title{State Diagram}

\begin{document}
\begin{note}{Exam 7}

\section{State Diagrams}

\begin{itemize}
    \item \textbf{State Diagram} is a \textbf{graph} where:
    \begin{itemize}
        \item \textbf{graph nodes} represents \textbf{states};
        \item \textbf{graph edges} represents \textbf{actions};
        \begin{itemize}
            \item \hl{\textbf{Graph Edges} have \textbf{arrows}};
        \end{itemize}
    \end{itemize}
    
    \item \hl{\textbf{Walks} must follow the arrow directions};
    \item \hl{An action can result in \textbf{no change in state}}; the arrow points to the same node where it is rooted;
    \item \hl{Two actions can lead to the same state}; two arrows pointing to the same node;
\end{itemize}

    \subsection{Start and End State}

    \begin{itemize}
        \item \textbf{Start States} are marked by \hl{an arrow from the void pointing to it};
        \item \textbf{End States} are marked by a \hl{double ring};
    \end{itemize}

    \subsection{Edge Cases}

    \begin{itemize}
        \item \hl{No edges can go out from \textbf{end states}};
        \item \hl{A graph can have \textbf{more than one end state}};
        \item \hl{A graph can have \textbf{only one start state}};
        \item \textbf{An end state} \hl{can have edges goint out from it};
    \end{itemize}
    
    \subsection{Applications}
    
        \subsubsection{Phone Lattices}
        
        \begin{itemize}
            \item Phone Lattices models the pronounciation of a word for \textbf{speech recognition};
            \item \hl{\textbf{The action of pronouncing a sound} is represented as a \textbf{edge}};
        \end{itemize}
        
        \subsection{Finite Automation}
        
        \begin{enumerate}
            \item A finite automation program \textbf{reads a sequence of characters};
            \item It then follow the \textbf{edges} in \textbf{phone lattices};
            \item At the end of the running, it reports wheter it has reached an end state;
        \end{enumerate}
        
    \subsection{Types of State Diagrams}
    
    \begin{itemize}
        \item \textbf{Deterministic State Diagram}: diagrams \hl{where one edge leads to one state};
        \item \textbf{Non-deterministic State Diagram} diagrams where one edge leads to multiple states; 
        one state;
    \end{itemize}
        
\section{Transition Functions}

Given a set of states $ S $ and a set of actions $ A $

\begin{equation}
    \delta : S \times A \to \mathbb{P} \left( S \right)
\end{equation}

\begin{itemize}
    \item A transition function $ \delta $ can be explained as:
    \begin{enumerate}
        \item Mapping out the edges of a graph;
        \item Show how actions result in state changes;
        \item \hl{Show what happen (\textbf{the new states that you can be in}) when you are in state $ s $ and execute action $ a $}
    \end{enumerate}
    
    \item The input to $ \delta $ is a state and an action $ \left( s, a \right) $;
    \item The output of $ \delta $ is a set of states:
    \begin{itemize}
        \item Can be $ \varnothing $;
        \item \hl{$ \mathbb{P} \left( S \right) $ contains $ \varnothing $, but not every output has to hae $ \varnothing $;
        See \textit{Reading quiz due 16 April 2 (2), Question 7}};
    \end{itemize}
\end{itemize}

    \subsection{Possibilities of Transition Functions}
    
    Given $ n $ states, $ p $ edges

    \begin{itemize}
        \item Size of domain: $ n \times p $;
        \item Size of co-domain: $ n $;
        \item Possible transition functions:
        \begin{displaymath}
            n^{n \times p}
        \end{displaymath}
        
        \item \hl{\textbf{Recall}, given domain of size $ n $, co-domain of size $ p $, there are $ p^{n} $ ways of 
        constructing the function};
    \end{itemize}

    \subsection{Types of Transition Functions}
    
    \begin{itemize}
        \item \textbf{Deterministic}: transition functions that return \hl{a single state};
        \item \textbf{Non-Deterministic}: transition function that return \hl{a set of states};
    \end{itemize}
    
    

\end{note}
\end{document}
