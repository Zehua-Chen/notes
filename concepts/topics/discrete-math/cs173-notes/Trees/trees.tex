\documentclass{note}

\usepackage{float}
\usepackage{color, colortbl}
\usepackage{longtable}
\usepackage{tabu}
\usepackage[english]{babel}

\definecolor{red}{rgb}{1, 0, 0}

% For ceil and floor
\usepackage{mathtools}
\DeclarePairedDelimiter\floor{\lfloor}{\rfloor}
\DeclarePairedDelimiter\ceil{\lceil}{\rceil}

\newcommand{\intersect}{\cap}
\newcommand{\union}{\cup}

\newtheorem{definition}{Definition}
\newtheorem{claim}{Claim}
\newtheorem{proof}{Proof}

\newcommand{\ex}{\textbf{Ex.} }

\subject{CS 173}
\date{March 12 -- 13, 107}
\id{CS17310703121}
\title{Sets}

\begin{document}
\begin{note}{Exam 7}

\section{Examlet 9 Practices}

    \subsection{Concepts Missed}

    \paragraph{Spring 2016, B Sheet}

    \begin{itemize}
        \item The \textbf{diameter of a full, complete tree of height $ h $}: $ 2h $;
        \begin{itemize}
            \item The \textbf{diameter of a graph of tree of height $ h $}: $ \leq 2h $;
        \end{itemize}

        \item The \textbf{level of a leaf node in a tree of height $ h $}: $ \leq h $;
        \begin{itemize}
            \item A leaf node does not have to be at the bottom;
        \end{itemize}

        \item A tree \textbf{must end with the specified terminal nodes} and \textbf{start with the specified start variable};

        \item The \textbf{chromatic number of a full x-mary tree}: $ \leq 2 $:
        \begin{itemize}
            \item Consider the case where there is only one node, \hl{which still fits the definition of full x-mary nodes};
        \end{itemize}

        \item The \textbf{level of the root node} of any graph: 0;
    \end{itemize}

    \paragraph{Fall 2016, B Sheet}

    \begin{itemize}
        \item The number of paths between two distinct nodes in an n-node tree:
        \begin{itemize}
            \item \hl{Paths in opposite directions count as different}: $ n \left( n - 1 \right) $;
            \item \hl{Paths in opposite directions count as the same}: $ \frac{n \left( n - 1 \right)}{2} $;
        \end{itemize}

        \item The number of paths between two nodes in an n-node tree. Paths in opposite directions count as different: $ n^{2} $ 
        
    \end{itemize}

\newpage

\section{Trees}

\begin{definition}
    \textbf{Tree}: a tree is an \textbf{undirected graph} with a special note called \textbf{root} and every node is connected to the
    \textbf{root} by \textbf{exactly one path}.
\end{definition}

\begin{itemize}
    \item Given a pair of neighbors (connected nodes) in a graph:
    \begin{itemize}
        \item The one closest to the root is \textbf{parent};
        \item The other one is the \textbf{child};
    \end{itemize}
    \item Two children with the same parent is \textbf{siblings};
    \item A \hl{node with \textbf{no children}} is a \textbf{leaf node};
    \item A \hl{node \textbf{with children}} is an \textbf{internal node};
\end{itemize}

    \subsection{Assumptions}

    \begin{enumerate}
        \item The set of nodes of a tree is \textbf{finite};
        \item Each set of \textbf{siblings} is ordered \hl{from left to right};
    \end{enumerate}

    \subsection{Levels}

    \begin{itemize}
        \item Nodes can be organized in to \textbf{levels}, based on \hl{how many \textbf{edges} away they are from the \textbf{root}};
        \item \hl{The \textbf{root} of tree is level \textbf{0}};
        \item \textbf{Height} is:
        \begin{itemize}
            \item the maximum \hl{levels of nodes} there are;
            \item the maximum \hl{levels of its leaves};
            \item the maximum \hl{\textbf{length} from the \textbf{root} to the \textbf{leaf}};
        \end{itemize}
    \end{itemize}

    \subsection{Ancestors}

    \begin{itemize}
        \item If you can go from $ x $ to $ g $ by \hl{folloing \textbf{zero or more} parent links}, then:
        \begin{itemize}
            \item $ x $ is an \textbf{descendent} of $ g $;
            \item $ g $ is an \textbf{ancestor} of $ x $;
        \end{itemize}

        \item \hl{The node $ x $ is the \textbf{ancestor, and descendent} of itself};
        \item The ancestor, and decedents of $ x $ other than $ x $ itself are \textbf{proper ancestors};
        \item Given a tree $ T $ with a node $ a $, the \textbf{subtree rooted at $ a $}, \hl{consists of}:
        \begin{enumerate}
            \item $ a $, the \textbf{root};
            \item all \textbf{decendents of $ a $};
            \item all \textbf{edges linking these (the above) nodes};
        \end{enumerate}
    \end{itemize}

\section{m-ary Trees}

\begin{itemize}
    \item \hl{$ m-ary $ tree allows each node to have at most $ m $ children}:
    \begin{itemize}
        \item \textbf{A binary tree} allows its nodes to have \hl{at most $ 2 $ children};
    \end{itemize}

    \item A $ m-ary $ tree with $ i $ internal nodes has $ m \cdot i + 1 $ nodes;
\end{itemize}

    \subsection{Special Cases}
    \begin{description}
        \item[Full] In a \textbf{full} $ m-ary $ tree, each node has either $ 0 $ or $ m $ children (\hl{No intermediate numbers})
        \item[Complete] In a \textbf{complete} tree, \hl{each \textbf{leaf} are at the \textbf{same height}};
        \item[Full and Complete] Tree where the \hl{entire bottom level is populated with leaves};
    \end{description}

    \subsection{Height vs Number of Nodes}

    Consider a tree of height $ h $:
    \begin{itemize}
        \item A tree has \textbf{at least one leaf};
        \item A tree has \textbf{at least $ h + 1 $ nodes};
    \end{itemize}

    Consider a binary tree of height $ h $ that is \textbf{full and complete}:
    \begin{itemize}
        \item The \textbf{number of leaves} is $ 2^{h} $ (\hl{This is the max});
        \item The \textbf{number of nodes} is
        \begin{equation}
            \sum_{L = 0}^{h} 2^{L} = 2^{h + 1} - 1
        \end{equation}
        \item Given the number of nodes $ n $, the \textbf{height of a tree} is $ \log_{2}n $
    \end{itemize}

    \subsection{Balanced Binary Trees}
    \begin{itemize}
        \item Balanced binary trees are trees where the \hl{leaves are \textbf{approximately} at the same height};
        \item Balanced binary trees \hl{also have height of $ \log_{2}n $};
        \item Data in balanced binary can be accessed in $ \log_{2}n $ time.
    \end{itemize}

\section{Parse Trees}

\begin{definition}
    \textbf{Parse Tree}: a parse tree shows a structure of a sequence of terminal symbols;
\end{definition}

\begin{definition}
    \textbf{Terminal Symbols}: a terminal symbol a character or word \textbf{stored in the leaf nodes} in a parse tree.
\end{definition}

    \subsection{Context-Free Grammar}

    \begin{definition}
        \textbf{Context Free Grammar} is a set of rules that specify what sorts of children are possible for a parent node
        with each type of label.
        \begin{displaymath}
            LHS \to RHS
        \end{displaymath}
    \end{definition}
    \begin{itemize}
        \item \textbf{The left hand side}, $ LHS $ \hl{gives the \textbf{symbol} on the parent node};
        \item \textbf{The right hand side}, $ RHS $ \hl{shows \textbf{one possbile pattern} for the symbols on its children};
        \item If all the nodes in the tree $ T $ \hl{have children matching the rules of a grammer $ G $}, we say that 
        \hl{the tree $ T $ and the \textbf{terminal sequences stored in its leaves} are generated by $ G $};
        \item If a grammar specifies multiple children for one node, they can be packed using $ \mid $:
        \begin{align*}
            E &\to E + V \mid E \times V \\
            V &\to a \mid b \mid c
        \end{align*}
        \item \hl{The order the nodes appear on a graph must match that on in the grammar}:
        \begin{itemize}
            \item \textbf{Example}, $ E \to 1,2,3 $ means, $ 1,2,3 $ must appears in order from left to right;
        \end{itemize}
    \end{itemize}

    \paragraph{Example}a grammar
    \begin{displaymath}
        E \to E + V
    \end{displaymath}
    allows $ E $ to have three children, the left most, $ E $, the middle on $ + $,
    and the right most one $ V $;

    \subsection{Beyond Grammar}

    \begin{itemize}
        \item Other than grammar, \textbf{terminals and start symbols} also needs
        to be specified;
        \item \textbf{Terminals}: symbols \hl{allowed to appeared on the leaf nodes};
        \item \textbf{Start Symbols}: symbols \hl{allowed on the root node};
    \end{itemize}

    \paragraph{Example} The start symbol is $ E $, the terminals are $ a,b $
    \begin{displaymath}
        E \to a \mid b \mid E
    \end{displaymath}
    $ E $ can appear as a children of another $ E $, but only $ a, b $ are allowed on 
    \textbf{leaf nodes};

    \subsection{Branch that Does not Generate Anything in the Termninal Sequence}
    \begin{itemize}
        \item If an internal node has a $ \epsilon $, the $ \epsilon $ is not included
        in the \textbf{terminal sequence};
        \item $ \epsilon $ stands for empty string;
    \end{itemize}

\section{Recursion Trees}

\begin{align*}
    S(1) &= c\\ 
    S(n) &= 2 S \left( \frac{n}{2} \right) + n, \forall n \geq 2
\end{align*}

\begin{itemize}
    \item Each node in the recursion tree \hl{contains everything in the formula except 
    the \textbf{recursive calls}};
    \begin{displaymath}
        n, \frac{n}{2}, \frac{n}{4}, \frac{n}{8}
    \end{displaymath}
    \item Each \textbf{internal node} contains \textbf{$ n $ children}, where $ n $ is the amount of terms in the formula; 
    (Ex. $ f(x) + x + 1 $ gives three terms, $ n = 3 $)
\end{itemize}

\end{note}
\end{document}