\documentclass{note}

\usepackage{float}
\usepackage{color, colortbl}
\usepackage{longtable}
\usepackage{tabu}
\usepackage{dirtytalk}
\usepackage[english]{babel}

\definecolor{red}{rgb}{1, 0, 0}

% For ceil and floor
\usepackage{mathtools}
\DeclarePairedDelimiter\floor{\lfloor}{\rfloor}
\DeclarePairedDelimiter\ceil{\lceil}{\rceil}

\newtheorem{definition}{Definition}

\subject{CS 173}
\date{April 22 -- 24, 107}
\id{CS17310704221}
\title{State Diagram}

\begin{document}
\begin{note}{Exam 12}

\section{Rationals and Reals}

\begin{itemize}
    \item Most real numbers are \textbf{irrational};
    \item \textbf{Rationals} is only a subset of reals;
\end{itemize}

    \subsection{Completeness}
    
    \begin{itemize}
        \item \textbf{Official Definition}: \say{It states that any subset of the reals with an upper bound has a smallest upper bound.}
        
        \item If a \hl{sequence of reals converges}, then the value the set converges to \hl{is also a real number};
        \begin{itemize}
            \item \hl{\textbf{Rationals} does not have this property};
        \end{itemize}
    \end{itemize}
    
\section{Cardinality}

\begin{itemize}
    \item \textbf{Definition of Same Cardinality}: two sets $ A $ and $ B $ have the same cardinality \textbf{if and only if}
    there is a \textbf{bijection from $ A $ to $ B $}\footnote{\textbf{Bijection}: a function that is both \textbf{one to one} 
    and \textbf{onto}};
    
    \item \textbf{Sets with the same cardinality as integers} are called \textbf{countably infinite}:
    \begin{itemize}
        \item A infinite set $ A $ is \hl{\textbf{countably infinite} if there is a \textbf{bijection} from 
        $ \mathbb{N} $ (or, $ \mathbb{Z} $) onto $ A $};
    \end{itemize} 
    
    \item \textbf{Countable}: both \hl{finite sets and sets that are countably infinite}; 
    \begin{itemize}
        \item All subsets of integers are countable;
        \item \hl{All subsets of \textbf{rational} numbers are countable};
    \end{itemize}
\end{itemize}

\section{Cantor Schroeder Bernstein Theorem}

\begin{equation}
    \left| A \right| \leq \left| B \right| \wedge \left| B \right| \leq \left| A \right| \to \left| A \right| = \left| B \right|
\end{equation}

\begin{itemize}
    \item \hl{If you can build a one to one function from in both direction, a bijection exists};
    \begin{itemize}
        \item For a one to one function to exist from $ A $ to $ B $, $ \left| A \right| \leq \left| B \right| $
    \end{itemize}
    
    \item If $ M $ is a collection of items, $ M * $ is the collection of \textbf{possible finite-length combinations of items} in $ M $;
    
    \item \textbf{Finite length strings of ASCII} characters are \textbf{countably infinite}:
    \begin{itemize}
        \item A one-to-one function that maps integers to strings can be constructed;
        \item A one-to-one function that maps strings to integers can be constructed as well: $ f(\epsilon) = 0  $
    \end{itemize}
\end{itemize}

\section{Countability}

    \subsection{Countable}
    
    \begin{itemize}
        \item Countable includes:
        \begin{enumerate}
            \item Sets that have \textbf{finite length};
            \item Sets that are \textbf{countably infinite};
        \end{enumerate}
        
        \item $ \varnothing $ is countable;
    \end{itemize}
    
    \subsection{Countability Infinite}
    
    Given a set $ A $, $ A $ \hl{is countably infinite if there is a \textbf{bijection} from $ \mathbb{N} $ or $ \mathbb{Z} $
    onto A};

    \subsection{$ \mathbb{P} (\mathbb{N} ) $ is not Countable}
    
    \begin{itemize}
        \item Use a technique named \textbf{diagonalization} developed by Georg Cantor; (\textbf{proof by contradiction});
        
        \item Given any set $ A $, $ \mathbb{P}(A) $ has a \textbf{strictly larger cardinality} than $ P $ does:
        \begin{equation}
            \forall A, \left| \mathbb{P}(A) \right| \geq \left| A \right| \wedge \left| A \right| \neq \left| \mathbb{P}(A) \right|
        \end{equation} 
        
        \item $ \left| \mathbb{P}(R) \right| = \left| R \right| $
    \end{itemize}

\section{Halting Problem}

\begin{itemize}
    \item It is impossible to build a program that reads through any program and determine if the input program eventually halt or 
    run forever;
    
\end{itemize}

    \subsection{Behavior of Computer Programs}
    
    \begin{enumerate}
        \item The program eventually halt, the trace is finite;
        \item The program loops, (return to previous states);
        \item The program keeps going, keeping consuming memory, \hl{rather than returning to a previous state};
        \begin{itemize}
            \item \hl{Even on a computer with no memory limitation, a program may not halt or return to a previous state eventually}.
        \end{itemize}
    \end{enumerate}

\section{Real Life Application of Countability}

\begin{itemize}
    \item The set of formula is \textbf{countable};
    \begin{itemize}
        \item A math equation is a finite string of characters;
    \end{itemize}
    
    \item The set of functions is \textbf{uncountable};
    
    \item The set of computer program is \textbf{countable};
    \begin{itemize}
        \item A computer program is just a finite state of ASCII characters;
    \end{itemize}
\end{itemize}

\end{note}
\end{document}
