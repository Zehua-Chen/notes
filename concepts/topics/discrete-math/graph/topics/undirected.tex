\chapter{Undirected Graph}

\section{Definitions}

  \paragraph{Unidirected Graph Definition} $ G = \left( V, E \right) $
  \begin{itemize}
    \item $ V $: a set of vertices
    \begin{itemize}
      \item $ \left| V \right| = n $
    \end{itemize}

    \item $ E $: a set of edges where each edge is of the form
    $ \left\{ u, v \right\}, u, v \in V, u \ne v $; $ u, v $ are endpoints of
    edges
    \begin{itemize}
      \item $ \left| E \right| = m $
    \end{itemize}
  \end{itemize}

  \paragraph{Multigraphs} Multigraphs are graphs that allow
  \begin{itemize}
    \item Loops
    \item Multi-edges
  \end{itemize}
  We assume a graph is a simple graph unless stated otherwise

\section{Representations}

  \subsection{Adjacency Matrix}

    Represent $ n $ vertices and $ m $ edges using an $ m \times n $
    adjacency matrix $ A $
    \begin{itemize}
      \item $ A\left[ i, j \right] = A\left[ j, i \right] = 1 $ if
      $ \left\{ i, j \right\} \in E $
      \item $ A\left[ i, j \right] = A\left[ j, i \right] = 0 $ if
      $ \left\{ i, j \right\} \notin E $
    \end{itemize}

    \paragraph{Advantages}
    \begin{itemize}
      \item Can check if $ \left\{ i, j \right\} \in E $ in
      $ O\left( n \right) $ time
    \end{itemize}

    \paragraph{Disadvantages}
    \begin{itemize}
      \item Need $ \Omega\left( n^{2} \right) $ space even if $ m < n^{2} $
    \end{itemize}

  \subsection{Adjacency List}

    For each $ u \in V $ store list
    $ \Adj\left( u \right) = \left\{ v | \left\{ u, v \right\} \in E \right\} $
    that is the neighbors of $ u $

    \paragraph{Advantages}
    \begin{itemize}
      \item Space is $ O\left( n + m \right) $
    \end{itemize}

    \paragraph{Disadvantages}
    \begin{itemize}
      \item Cannot check if $ \left\{ i, j \right\} \in E $ in
      $ O\left( 1 \right) $ time
      \begin{itemize}
        \item Plain list gives $ O\left( m \right) $ time
        \item Sorted list can give $ O\left( \log n \right) $ time
        \item Hashed list can give $ O\left( 1 \right) $ time
      \end{itemize}
    \end{itemize}

\section{Connectivity}

  \begin{itemize}
    \item A single vertex is a cycle, rather it is a \textbf{tour}
  \end{itemize}

  \subsection{Path}

    A sequence of distinct vertices $ v_{1} ... v_{k} $ such that
    $ \left\{ v_{1}, v_{i + 1} \right\} \in E $ for $ 1 \le i \le k - 1 $

    \begin{itemize}
      \item The path is from $ v_{i} $ to $ v_{k} $
      \item Length of hte path is the number of edges, $ k - 1 $
      \item A single vertex gives a path length of $ 0 $
    \end{itemize}

  \subsection{Walk}

    A walk is a traversal of a graph.

    \begin{itemize}
      \item Walks include edges and vertices
      \item Edges and vertices can be repeated
      \item Length of the walk is the numebr of edges
    \end{itemize}

  \subsection{Connected}

    $ u $ is connected to $ v $ if there is a path from $ u $ to $ v $

  \subsection{Connected Component}

    The connected component of $ u $, $ \cont\left( u \right) $, is the
    set of all vertices connected to $ u $.

    \begin{itemize}
      \item A graph is connected if there is only one component
    \end{itemize}
