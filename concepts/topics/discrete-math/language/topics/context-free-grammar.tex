\chapter{Context Free Grammar/Languages}

\section{Definition}

  \begin{align}
    CFG &= \left( V, T, P, S \right) \\
    V &= \text{finite set of non-terminal symbols} \\
    T &= \text{terminal symbols} \\
    P &= \text{productions} \\
    P &= V \times \left( V \cup T \right)^{*} \\
    S &\in V
  \end{align}

  Productions are of the form

  \begin{align}
    A &\to \alpha \\
    A &\in V \\
    \alpha &\in \left( V \cup T \right)^{*}
  \end{align}

  \subsection{Conventions}

    \begin{align}
      a, b, c ... &\in T \\
      A, B, C ... &\in V \\
      u, v, w, x, y ... &\in T^{*} \\
      \alpha, \beta, \gamma, ... &\in \left( V \cup T \right)^{*} \\
      X, Y, Z &\in \left( V \cup T \right)
    \end{align}

\section{Derives}

  Given a $ G $, for strings $ \alpha_{1}, \alpha_{2} \in \left( U \cup T \right)^{*} $,
  we say $ \alpha_{1} \leadsto_{G} \alpha_{2} $ (derives) if ther exists
  strings $ \beta, \gamma, \delta \in \left( V \cup T \right)^{*} $ such that

  \begin{align}
    \left( A \to \gamma \right) &\in P \\
    \alpha_{1} &= \beta A \delta \\
    \alpha_{2} &= \beta \gamma \delta
  \end{align}

  \subsection{More Than One Step}

    For integers $ k \ge 0 $, $ \alpha_{1} \leadsto^{k} \alpha_{2} $ is
    defined as

    \begin{align}
      \alpha_{1} &\leadsto^{0} \alpha_{2} \text{ if } \alpha_{1} = \alpha_{2} \\
      \alpha_{1} &\leadsto^{k} \alpha_{2} \text{ if } \alpha_{1} = \beta_{1}, \beta_{1} \leadsto^{k - 1} \alpha_{2}
    \end{align}

    $ \leadsto^{*} $ is the reflexive and transitive closure of $ \leadsto $

    \begin{align*}
      \alpha_{1} \leadsto^{*} \alpha_{2} \text{ if } \alpha_{1} \leadsto^{k} \alpha_{2}, k \ge 0
    \end{align*}

\section{Context Free Languages}

  The language $ L $ generated by $ G $ is denoted by

  \begin{align}
    L\left( G \right) &= \left\{ w \in T^{*} | S \leadsto^{*} w \right\}
  \end{align}

  A langauage is \textbf{context free} if it is generated by
  a context free grammar; that is there is a CFG $ G $ such that

  \begin{align}
    L &= L\left( G \right)
  \end{align}

\section{Proof}

  Prove
  \begin{equation*}
    L\left( G \right) = L
  \end{equation*}

  \begin{enumerate}
    \item Direction 1
    \begin{align*}
      L \left( G \right) &\subseteq L
    \end{align*}

    if $ S \leadsto^{*} w $, then ...

    \item Direction 2
    \begin{align*}
      L &\subseteq \left( G \right)
    \end{align*}

    If $ S \leadsto^{*} w $, then ...
  \end{enumerate}

\section{Closure Properties}

  \begin{itemize}
    \item \textbf{Closed}
    \begin{itemize}
      \item Given $ L_{1}, L_{2} $ are CFLs $ L_{1} \cup L_{2} $ is a CFL
      \item Concatination
      \item Kleene star
      \item If $ L_{1} $ is a CFL and $ L_{2} $ is regular then
      $ L_{1} \cap L_{2} $ is a CFL
    \end{itemize}

    \item \textbf{Not closed}
    \begin{itemize}
      \item Complement
      \item Intersection
    \end{itemize}
  \end{itemize}

\section{Trees}

  Trees are used to represent derivations

  \begin{itemize}
    \item Root labeled $ S $
    \item Non terminals at each internal node of tree
    \item Terminal at leaves
    \item Children of internal node indicate how non terminal was expanded
    using a production rule
  \end{itemize}

\section{Ambiguity}

  \begin{itemize}
    \item If a string has two parse trees, it is ambiguous
    \item If for a CFG $ G $ no ambiguous strings exists, the CFG is unambiguous
    \item A CFL $ L $ is inheritantly ambiguous if ther eis no unambiguous
    CFG $ G $ such that $ L = L \left( G \right) $
    \begin{itemize}
      \item
    \end{itemize}
  \end{itemize}

  \subsection{Solution}

    \begin{itemize}
      \item Add a variable (ex. make it left or right derived)
    \end{itemize}

\section{Normal Forms}

  \begin{itemize}
    \item Way to restrict form of production rule
    \item Advantages
    \begin{itemize}
      \item Linear parsing
      \item Simpler and more convenient algorithms and proofs
    \end{itemize}

    \item Variants
    \begin{itemize}
      \item Chomsky
      \item Greibach
    \end{itemize}
  \end{itemize}

  \subsection{Chomsky Normal Form (CNF)}

    \begin{align}
      A &\to BC \\
      A &\to a
    \end{align}

    If $ \epsilon \in L $ then $ S \to \epsilon $ is also allowed

    \begin{itemize}
      \item Every CFG can be converted to CNF via an efficient algorithm
      \item Advantage
      \begin{itemize}
        \item Parse tree of constant degree
      \end{itemize}
    \end{itemize}

  \subsection{Greiback Normal Form (GNF)}

    \begin{align}
      A &\to a \beta
    \end{align}

    \begin{itemize}
      \item All CFLs without $ \epsilon $ have a grammar in GNF
      \item \textbf{Advantages}
      \begin{itemize}
        \item Efficient algorithm
        \item Every derivation adds exactly one terminal
      \end{itemize}
    \end{itemize}

\section{From DFA}

  \begin{enumerate}
    \item $ T = \left\{ S_{q} | q \in Q \right\} $
    \item $ S = S_{s} $
    \item $ S_{q} \to a S_{q'} b $
  \end{enumerate}
