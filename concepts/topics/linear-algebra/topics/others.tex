\chapter{Others}

\section{Singular Value Decomposition}

  Given that for matrices $ A, B $ must be of sizes
  $ \left( ?, a \right), \left( a, ? \right) $. We can derive that the
  sizes of the matrices $ U, \sum, V $ must have the pattern
  \begin{align*}
    \left( m, n \right)
      &= \left( m, a \right) \times \left( a, b \right) \times \left( b, n \right) \\
      &= \left( m, a \right) \times \left( a, b \right) \times \left( n, b \right)^{T}
  \end{align*}

  Since we know that $ U, V $ are orthogonal matrices, we know that they
  are square matrices.

  \begin{align*}
    \left( m, n \right)
      &= \left( m, m \right) \times \left( a, b \right) \times \left( n, n \right)^{T}
  \end{align*}

  From this, we can know that $ \sum $ must be of $ \left( m, n \right) $.

  In conclusion, $ U $ has size $ \left( m, m \right) $, $ \sum $
  has size $ \left( m, n \right) $ and $ V $ has size $ \left( n, n \right) $.

  \subsection{Inverse}

    Let $ M^{-1} = A B C $
    \begin{align*}
      M \times M^{-1} &= \left( A B C \right) U \Sigma V^{T} \\
      I &= A B \left( C U \right) \Sigma V^{T} \\
    \end{align*}

    If $ C = U^{T} $, then $ C U^{T} = I $, as $ U $ is an orthogonal matrix.

    \begin{align*}
      I &= A \left( B \Sigma \right) V^{T}
    \end{align*}

    Since $ \Sigma $ is not an orthogonal matrix, $ B $ must be $ \Sigma^{-1} $
    for $ B \Sigma = I $.

    \begin{align*}
      I &= A V^{T}
    \end{align*}

    If $ A = V $, then $ A V^{T} = I $, as $ V $ is an orthogonal matrix.

    In conclusion,

    \begin{align*}
      M^{-1} = V \Sigma^{-1} U^{T}
    \end{align*}
