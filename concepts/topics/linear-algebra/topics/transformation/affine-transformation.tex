\section{Affine Transformations}

\textbf{An affine transformation is the sum of a linear transformation and a
constant vector};

\begin{itemize}
  \item Linear transformation preserves the origin; affine transformation do
  not need to preserve the origin
  \item Affine transformation map the origin to a new position
  \item Most affine transformations do not commute
  \item Some (combinations of) transformations commute
  \footnote{if you change the order of transformation, the result remains the
  same}
  \begin{itemize}
    \item A rotation and a uniform scaling
    \item Two translations
    \item Two rotations around the same axis
  \end{itemize}

  \item All the transformations here are done using \textbf{homogeneous
  coordinates}
  \begin{equation}
    \left( x, y, z, w \right) \Leftrightarrow
    \left( \frac{x}{w}, \frac{y}{w}, \frac{z}{w} \right)
  \end{equation}
  \begin{itemize}
    \item Cartesian coordinates ($ \left( \frac{x}{w}, \frac{y}{w}, \frac{z}{w} \right) $)
    are used in Euclidean Geometry
    \item Homogeneous coordinates ($ \left( x, y, z, w \right) $) are used in
    Projective Geometry; $ w = 0 $ corresponds to a point at infinity or a
    direction, instead of a position
    \item Exists to make most operations doable as affine transformations
  \end{itemize}
\end{itemize}

\subsection{Translation}

  \begin{align}
    T\left( x, y, z \right) &=
    \begin{bmatrix}
      1 & 0 & 0 & x \\
      0 & 1 & 0 & y \\
      0 & 0 & 1 & z \\
      0 & 0 & 0 & 1
    \end{bmatrix}
    \begin{bmatrix}
      x \\
      y \\
      z \\
      1
    \end{bmatrix} \\
    T\left( x, y, z \right)^{-1} &=
    \begin{bmatrix}
      1 & 0 & 0 & -x \\
      0 & 1 & 0 & -y \\
      0 & 0 & 1 & -z \\
      0 & 0 & 0 & 1
    \end{bmatrix}
    \begin{bmatrix}
      x \\
      y \\
      z \\
      1
    \end{bmatrix}
  \end{align}

\subsection{Scaling}

  \begin{align}
    S\left( x, y, z \right) &=
    \begin{bmatrix}
      x & 0 & 0 & 0 \\
      0 & y & 0 & 0 \\
      0 & 0 & z & 0 \\
      0 & 0 & 0 & 1
    \end{bmatrix}
    \begin{bmatrix}
      x \\
      y \\
      z \\
      1
    \end{bmatrix} \label{eq: affine-transformation-scale-bigger} \\
    S\left( x, y, z \right)^{-1} &=
    \begin{bmatrix}
      \frac{1}{x} & 0 & 0 & 0 \\
      0 & \frac{1}{y} & 0 & 0 \\
      0 & 0 & \frac{1}{z} & 0 \\
      0 & 0 & 0 & 1
    \end{bmatrix}
    \begin{bmatrix}
      x \\
      y \\
      z \\
      1
    \end{bmatrix}
  \end{align}

  Use $ -1 $ in any of $ x, y, z $ in equation
  \ref{eq: affine-transformation-scale-bigger} would give an inversion matrix
  around a certian axis

\subsection{Shear}

  \begin{equation}
    H_{xz}\left( s \right) =
    \begin{bmatrix}
      1 & 0 & s & 0 \\
      0 & 1 & 0 & 0 \\
      0 & 0 & 1 & 0 \\
      0 & 0 & 0 & 1
    \end{bmatrix}
  \end{equation}

  \begin{itemize}
    \item Affine transformation
  \end{itemize}

\subsubsection{Rotation}

  \begin{align}
    R_{x}\left( \theta \right) &=
    \begin{bmatrix}
      1 & 0 & 0 & 0 \\
      0 & \cos\theta & -\sin\theta & 0 \\
      0 & \sin\theta & \cos\theta & 0 \\
      0 & 0 & 0 & 1
    \end{bmatrix}
    \begin{bmatrix}
      x \\
      y \\
      z \\
      1
    \end{bmatrix} \\
    R_{y}\left( \theta \right) &=
    \begin{bmatrix}
      \cos\theta & 0 & \sin\theta & 0 \\
      0 & 1 & 0 & 0 \\
      -\sin\theta & 0 & \cos\theta & 0 \\
      0 & 0 & 0 & 1
    \end{bmatrix}
    \begin{bmatrix}
      x \\
      y \\
      z \\
      1
    \end{bmatrix} \\
    R_{z}\left( \theta \right) &=
    \begin{bmatrix}
      \cos\theta & -\sin\theta & 0 & 0 \\
      \sin\theta & \cos\theta & 0 & 0 \\
      0 & 0 & 1 & 0 \\
      0 & 0 & 0 & 1
    \end{bmatrix}
    \begin{bmatrix}
      x \\
      y \\
      z \\
      1
    \end{bmatrix} \\
    R^{-1} &= R^{T}
  \end{align}

  \begin{align*}
    R_{u}\left( u, \theta \right) &=
    \begin{bmatrix}
      % row 1
      \cos\theta + u_{x}^{2}\left( 1 - \cos\theta \right)
      & u_{x} u_{y} \left( 1 - \cos\theta \right) - u_{z} \sin\theta
      & u_{x} u_{z} \left( 1 - \cos\theta \right) + u_{y} \sin\theta \\
      % row 2
      u_{y} u_{x} \left( 1 - \cos\theta \right) + u_{z} \sin\theta
      & \cos\theta + u_{y}^{2}\left( 1 - \cos\theta \right)
      & u_{y} u_{z} \left( 1 - \cos\theta \right) - u_{x} \sin\theta \\
      % row 3
      u_{z} u_{x} \left( 1 - \cos\theta \right) - u_{y} \sin\theta
      & u_{z} u_{y} \left( 1 - \cos\theta \right) + u_{x} \sin\theta
      & \cos\theta + u_{z}^{2}\left( 1 - \cos\theta \right)
    \end{bmatrix} \\
    &\times
    \begin{bmatrix}
      x \\
      y \\
      z \\
      1
    \end{bmatrix}
  \end{align*}

  \subsubsection{Rotation Around an Object's Center}

    \begin{enumerate}
      \item Translate to the origin
      \item Rotate
      \item Translate back using the inverse
    \end{enumerate}

\subsection{Composing Transformations}

  \begin{itemize}
    \item Transformations are applied \textbf{right to left}
    \begin{equation*}
      O = A \times B \times M
    \end{equation*}
    Given a coordinate $ M $, $ B $ is applied first, followed by $ A $. Note
    that matrix multiplications still works from \textbf{left to right}
  \end{itemize}

  \subsubsection{Inverse}

    \begin{equation}
      \left( T_{1} T_{2} T_{3} ... \right) ^{-1}
      = \left( ... T_{3}^{-1} T_{2}^{-1} T_{1}^{-1} \right) ^{-1}
    \end{equation}

    \begin{itemize}
      \item Given a transformation $ t $ on an object, applying the inverse of
      $ t $ on the origin would produce the same result
    \end{itemize}

\subsection{Rigid Body Transformations}

  \begin{itemize}
    \item Non rigidbody transformations
    \begin{itemize}
      \item Scale
      \item Shear
    \end{itemize}
  \end{itemize}
