\chapter{Systems}

\begin{itemize}
  \item \textbf{\Gls{system}}: \glsdesc{system}
  \item A system with no memory only needs to know its input at one time
  $ t $ to produce the output of $ t $
  \item A system with memory need inputs at more than one moment in the past
  to produce its outputs
  \item \textbf{Parallel System}: the input is processed by two systems
  concurrently. The output is the the sum of the outputs by two filters
  \item \textbf{Chain System}: the output of one system is the input
  of another system
  \item \textbf{FIR}: finite impulse response
  \item \textbf{IIR}: infinite impulse response
\end{itemize}

\section{Causality}

  A system is causal if output at time $ n $ only depends on input and output
  up to time $ n $

\section{Linear Systems}

  \begin{equation}
    y\left( t \right) =
      \sum_{j = 0}^{M} b_{j} x\left( t - j \right)
      - \sum_{i = 1}^{N} a_{i} y\left( t - i \right)
  \end{equation}

  The output of a linear system is the sum of scaled inputs and
  scaled previous outputs.

  \begin{itemize}
    \item $ b_{x} $ scale of inputs
    \item $ a_{x} $ scale of outputs
    \item $ i $ starts from $ 1 $ because $ y\left( t \right) $ is on the left
    of the equation
    \item A linear system may only be linear over a limited range of input;
    ex. auditory system
    \item Techniques developed for linear systems can help understand
    non-linear system; ex. filter
  \end{itemize}

  \subsection{Linearity}

    Linearity is \gls{homogeneity} and \gls{additivity} combined

    \begin{itemize}
      \item \textbf{\Gls{homogeneity}}: \glsdesc{homogeneity}
      \item \textbf{\Gls{additivity}}: \glsdesc{additivity}
    \end{itemize}

\section{Linear Time Invariant}

  \textbf{\Gls{linear-time-invariant}}: \glsdesc{linear-time-invariant}

  \begin{itemize}
    \item LTI system is an ideal model of a subset of systems; it is an
    approximation to real situation
    \item Some parts of a non-linear system can be modeled as a LTI system,
    ex. vocal tract
    \item A time-variant system may be treated as a LTI system in a short
    period of time
    \item \textbf{Homogeneity and time invariance}: sinusoids are only changed
    in amplitude and phase as they are processed through LTI systems
    \item \textbf{Additivity and fourier theorem}: if the responses of an
    LTI system to sinusoids of all frequencies are known, the system outupt
    to any input signal can be expressed as a sum of sinusoids with different
    frequencies, amplitudes and phases
    \begin{itemize}
      \item Fourier transform can break signals into sinusoid signals
    \end{itemize}
  \end{itemize}

  \subsection{Pre-Emphasis}

    \begin{equation}
      y\left( t \right) = x\left( t \right) - \alpha x\left( t - 1\right)
    \end{equation}

    \begin{itemize}
      \item Pre-emphasis is an example of a filter
    \end{itemize}

\section{Impulse Response}

  IR tells how a system reacts to \textbf{any stimulus}

  \begin{itemize}
    \item Impulse response is time domain
  \end{itemize}

\section{Convolution}

  \begin{equation}
    y\left( t \right)
      = x\left( t \right) \otimes h\left( t \right)
      = h\left( t \right) \otimes x\left( t \right)
  \end{equation}

  The output of any linear system is given by the convolution between
  the \textbf{input} with the \textbf{response of the system to an impulse,
  aka. impulse response}.

  \begin{itemize}
    \item Convolution is hard to visualize
    \item The convolution of a signal can be visualized in frequency domain
    using the fourier transform (section \ref{sec: fourier transform}) of the
    signal and the impulse response
    \item Convolution is the standard model of speech production
    \item Convolution is the backbone of many types of signal processing
    \begin{enumerate}
      \item Image processing
      \item Signal filtering
      \item Audio processing
      \item Machine learning/AI
    \end{enumerate}
  \end{itemize}

\section{Response}

  \subsection{Frequency Response}

    A frequency response is a quantitative measure of the output spectrum
    of a system or a device in response to a stimulus

    \begin{itemize}
      \item Composed of complex numbers
      \item Frequency domain
    \end{itemize}

  \subsection{Amplitude Response}

    The magnitude of a frequency response is amplitude response and gives
    the filter gain at each frequency

    \begin{equation}
      \dB = 20 \log_{10} \left( A \right)
    \end{equation}

    Where $ A $ is the amplitude response

  \subsection{Impulse Response}

    \begin{itemize}
      \item Frequency response in time domain
    \end{itemize}

\section{Filters}

  \begin{itemize}
    \item Low-pass filters attenuate high-frequency sinusoids more tahn
    low-frequency sinusoids
    \item \textbf{Cutoff frequency}: the frequency location above, below which
    the filter passes all sinusoids
    \item \textbf{Pass-band}: the frequency region over which the
    sinusoids can pass the system
    \begin{itemize}
      \item $ > 0 dB $
    \end{itemize}
    \item \textbf{Stop-band}: the frequency region over which the
    sinusoids are attenuated
    \begin{itemize}
      \item $ < 0 dB $
    \end{itemize}
  \end{itemize}

  \subsection{Chain System}

    \begin{align}
      c\left( x \right) &= b \left( a\left( x\right) \right) \\
      R^{c}\left( f \right) &= R^{b}\left( f \right) \cdot R^{a}\left( f \right) \\
      H^{c}\left( t \right) &= H^{b}\left( t \right) \cdot H^{a}\left( t \right)
    \end{align}

    \begin{itemize}
      \item In time-domain: C can be produced by performing convolution on
      $ a, b $
      \item In frequency-domain: C can be produced by $ a \times b $
      \item $ R $: frequency response
      \item $ H $: impulse response
    \end{itemize}

  \subsection{Parallel System}

    \begin{align}
      c\left( x \right) &= b \left( x \right) + a\left( x\right) \\
      R^{c}\left( f \right) &= R^{b}\left( f \right) + R^{a}\left( f \right) \\
      H^{c}\left( t \right) &= H^{b}\left( t \right) \otimes H^{a}\left( t \right)
    \end{align}

  \subsection{Band Pass}

    A band-pass filter can be generated from a low-pass and a high-pass filter
    by multipling the frequency responses of those two filters,
    which can also be done by convolving the two filters in time domain.

    \begin{itemize}
      \item Make sure the size of the result of convolution is the same as
      the size of each of the two child filters
    \end{itemize}
