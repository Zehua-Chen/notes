\chapter{Signals}

\section{Signals}

  \begin{itemize}
    \item \textbf{\Gls{signal}}: \glsdesc{signal}
    \begin{itemize}
      \item Some signals can be described mathmetically
      \item Some signals are too complicated to be described mathmetically
      \item Signals can be 1D (audio), 2D (image), 3D (movie)
    \end{itemize}

    \item \textbf{\Gls{continuous-signal}}: \glsdesc{continuous-signal}
    \begin{itemize}
      \item We need to be able to convert analogue signals to continuous signals
      for processing
    \end{itemize}
  \end{itemize}

  \subsection{Analogue vs Digital Signal}

    \begin{itemize}
      \item \textbf{Analogue Signal}
      \begin{enumerate}
        \item \textbf{Pro}: fast, well-behaved and degrades gracefully
        \item \textbf{Con}: incompatible with computers
        \item \textbf{Con}: needs special hardware
        \item \textbf{Con}: practical limits to signal processing
      \end{enumerate}

      \item \textbf{Digital Signal}
      \begin{enumerate}
        \item \textbf{Pro}: unlimited processing without specialized hardware
        \item \textbf{Pro}: can employ error correcting codes
        \item \textbf{Con}: needs care in acquisition
        \item \textbf{Con}: slow to process and potential loss of information
      \end{enumerate}
    \end{itemize}

  \subsection{Workflow of DSP Systems}

    \begin{enumerate}
      \item \textbf{A/D Converter}: convert analogue signal to digital signal
      \item \textbf{Disgital Signal Processor}: take in digital input and
      produce digital output
      \item \textbf{D/A Converter}: convert digital signal to analogue signal
    \end{enumerate}

  \subsection{Types of Operations}

    \begin{enumerate}
      \item \textbf{Time-domain}: scaling, shifting, addition
      \item \textbf{Correlation}: comparing one reference signal with others
      to determine similarity
      \item \textbf{Digital filtering}: let certain band of specific frequencies
      pass while blocking others
      \item \textbf{Modulation and demodulation}: amplitude modulation,
      frequency modulation, phase modulation
      \item \textbf{Discrete transformation}: represent a discrete-time signal
      in frequency domain
    \end{enumerate}

  \subsection{Three Fundamental Ideas of DSP}

    \begin{enumerate}
      \item Any signal can be represented using a finite number of discrete
      data points over time
      \item Any signal can be broken down into a stream of bits that can
      be stored, generated and manipulated by a computer
      \item Any practical signals can be represented as sums of sinusoids
      (or complex exponentials)
    \end{enumerate}

\section{Properties of Signals}

  \begin{itemize}
    \item \textbf{\Gls{period}} ($ T $): \glsdesc{period}
    \begin{itemize}
      \item Speech waveforms are not periodic
    \end{itemize}

    \item \textbf{\Gls{frequency}} ($ f $): \glsdesc{frequency}
    \begin{align}
      f &= \frac{1}{T} \\
      \omega &= 2 \pi f \\
    \end{align}
    \begin{align*}
      y\left( t \right) &= A \sin\left( 2\pi f t + \phi \right) \\
      y\left( t \right) &= A \sin\left( w t + \phi \right) \\
    \end{align*}
    \begin{itemize}
      \item $ \omega $: angular frequency; (rad/s)
      \item $ f $: linear frequency; (Hz)
    \end{itemize}

    \item \textbf{\Gls{phase}} ($ \phi $): \glsdesc{phase}
    \begin{align}
      d &= \frac{360 \phi}{2 \pi} \\
      \phi &= \frac{2 \pi d}{360}
    \end{align}
    \begin{itemize}
      \item $ \phi $: phase in radians
      \item $ d $: phase in angle
    \end{itemize}
  \end{itemize}

  \subsection{Properties of Continuous Signal}

    \begin{equation*}
      x\left( t \right) = A \sin\left( 2\pi f t + \phi \right)
    \end{equation*}

    \begin{enumerate}
      \item Inputs to discrete signals are floating points
      \begin{itemize}
        \item Ex. if $ x \in \left[-1, 1\right] $, then
        there are infinite number of $ x $
      \end{itemize}

      \item For every fixed value of $ f $, $ x\left( t \right) $ is periodic
      with period being $ T = \frac{1}{f} $
      \item Continuous signals with distinct equations are themselves distinct
      \item Increasing $ f $ results in an increase in the rate of oscillation,
      as $ t $ is continuous, we can increase $ f $ without limit
    \end{enumerate}

  \subsection{Properties of Discrete Signal}

    \begin{equation*}
      x\left( n \right) = A \sin\left( \omega' n + \phi \right)
    \end{equation*}

    \begin{align}
      \omega' &= 2\pi f' \\
      f' &= \frac{f}{f_{s}}
    \end{align}

    $ f' $ is cycles per sample and $ f_{s} $ is the sampling frequency

    \begin{enumerate}
      \item Inputs to discrete signals are integers
      \begin{itemize}
        \item Ex. if $ n \in \left[-1, 1\right] $, then
        $ n = \left\{ -1, 0, 1 \right\} $
      \end{itemize}

      \item A discrete time signal is periodic if $ f' $ is a rational number
      \begin{itemize}
        \item Fundamental period $ N $
      \end{itemize}

      \item Discrete time signals whose frequencies are separated by an
      integer multiple of $ 2 \pi $ are identical
      \item The highest rate of oscillation is when $ \omega = \pi $ or
      $ f = \frac{1}{2} $
      \item \textbf{Aliasing}: if the angular frequency of a discrete time
      signal increases from $ \pi $ to $ 2\pi $, its rate of oscillation
      will decrease
    \end{enumerate}

\section{Digitalisation}

  DSP requires a signal to be \textbf{sampled} in time and \textbf{quantised}
  in amplitude.

  \begin{itemize}
    \item \textbf{\Gls{sampling}}: \glsdesc{sampling}
    \item \textbf{\Gls{quantisation}}: \glsdesc{quantisation}
  \end{itemize}

  \subsection{Sampling}

    \begin{equation}
      t = n \Delta_{s} = \frac{n}{f_{s}}
    \end{equation}

    \begin{itemize}
      \item $ t $: continuous time in seconds
      \item $ f_{s} $: sampling frequency
      \item $ n $: number of seconds
      \item $ \Delta_{s} $: the sampling period
    \end{itemize}

  \subsection{Quantisation}

    \begin{itemize}
      \item \textbf{\Gls{quantisation-level}}: \glsdesc{quantisation-level}
      \begin{itemize}
        \item The number of bits determines the number of levels; given $ n $
        bits, there will be $ 2^{n} $ quantisation levels
        \item 16 bits is typically used
      \end{itemize}

      \item \textbf{\Gls{quantisation-step}}: \glsdesc{quantisation-step}
      \item \textbf{\Gls{quantisation-error}}: \glsdesc{quantisation-error}
      \begin{equation}
        e_{q}\left( n \right) = x_{q}\left( n \right) - x\left( n \right)
      \end{equation}
      \begin{itemize}
        \item There will alwyas be \gls{quantisation-error}
        \item It is always possible to talk about the signal to noise ratio
        (SNR) of a clean digital signal
        \item Distortion, i.e. clipping, makes \gls{quantisation-error}
        difficult to handle
      \end{itemize}
    \end{itemize}

    \begin{equation}
      \rms_{e} = \sqrt{\frac{\sum_{1}{N} e_{q}^{2}\left( n \right)}{N}}
    \end{equation}

\section{Nyquist Frequency/Limit}

  A signal can only be recovered from a sampled signal if the signal is
  sampled at $ f_{s} $ where

  \begin{align}
    f_{s} &> 2 f \\
    \frac{f_{s}}{2} &> f
  \end{align}

  $ \frac{f_{s}}{2} $ is called the \textbf{Nyquist frequency/limit}

  \paragraph{Example}
  To sample a 100 Hz sine wave, an $ f_{s} > 200 $ is required

  \subsection{Aliasing}

    \begin{enumerate}
      \item Given a $ f_{s} $ a sine wave at $ f $ is indistinguishable
      from a sine wave at frequency $ f - f_{s} $
      \item Given a $ f_{s} $ and an integer $ k $, a sine wave at
      $ f $, where $ 0 < f < \frac{f_{s}}{2} $ is indistinguishable
      from a sine wave at frequency $ f + k \times f_{s} $ after being sampled
      by $ f_{s} $
    \end{enumerate}
