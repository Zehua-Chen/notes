\chapter{Fourier Transform}
\label{sec: fourier transform}

Given a soup, fourier transform finds the recipe by running the soup through
filters.

\begin{itemize}
  \item Fourier transform translates from time domain to frequency domain
  \item Inverted Fourier transform translates from frequency domain to time
  domain
\end{itemize}

\section{Fourier Series}

  \begin{equation}
    x\left( t \right) = A_{0}
      + \sum_{p = 1}^{\infty} A_{p} \cos\left( p w_{0} t - \phi_{p} \right)
  \end{equation}

  \begin{itemize}
    \item $ A_{0} $: direct current
    \item $ p w_{0} $: harmonics of $ w_{0} $ when $ p \ge 2 $
    \item The output of fourier transform stretches from 0 to $ fs $ Hz
    \begin{itemize}
      \item Each point on the x-axis reprsent a distance of $ \frac{fs}{N} $ Hz
      from $ 0 $
      \item Frequency resolution is porportional to $ \frac{1}{N} $
    \end{itemize}
  \end{itemize}

  \subsection{Finite Fourier Series Approximation}

    \begin{equation}
      x\left( t \right) = A_{0}
        + \sum_{p = 1}^{P} A_{p} \cos\left( p w_{0} t - \phi_{p} \right)
    \end{equation}

  \subsection{General Form}

    \begin{align}
      A_{p} &= \sqrt{a_{p}^{2} + b_{p}^{2}} \\
      \phi_{p} &= \tan^{-1} \left( \frac{b_{p}}{a_{p}} \right)
    \end{align}

    \begin{equation}
      x\left( t \right) = a_{0}
        + \sum_{p = 1}^{\infty}
        \left(
          a_{p} \cos\left( p w_{0} t \right)
          + b_{p} \sin\left( p w_{0} t \right)
        \right)
    \end{equation}

    \begin{itemize}
      \item Fourier coefficients
    \end{itemize}

\section{Discrete Fourier Transform}

  \begin{align}
    X\left( k \right) &=
      \frac{1}{N} \sum_{n = 0}^{N - 1} x\left( n \right)
      \cdot e^{-i \varphi} \\
    \varphi &= \frac{2 \pi k n}{N} \\
    X\left( k \right) &=
      \frac{1}{N} \sum_{n = 0}^{N - 1} x\left( n \right)
      \cdot \left( \cos\left( \phi \right) - i \sin\left( \phi \right) \right) \\
  \end{align}

  \begin{itemize}
    \item $ x\left( n \right) $: time domain signal of length $ N $
    \item $ X\left( k \right) $: amplitude and phase info at frequency bin $ k $
    \begin{itemize}
      \item For $ k = 0 $, $ X\left( 0 \right) $ is simply the mean of signal
      samples
    \end{itemize}

    \item \textbf{Input}: array of time domain samples
    \item \textbf{Output}: array of frequency domain samples
  \end{itemize}

\section{Fast Fourier Transform}

  \begin{itemize}
    \item FFT is an algorithm for the efficient calculation of DFT
    \item Fast Fourier Transform is an implementation of Fourier Transform
  \end{itemize}

\section{Spectogram}

  \begin{itemize}
    \item x-axis: time
    \item y-axis: frequency
    \item z-axis, values at $ \left( x, y \right) $: the amplitude of a
    particular frequency at a particular time is represented by the intensity
    or color of each point in the image
  \end{itemize}
