\chapter{Query Evaluation}

  \begin{itemize}
    \item \emph{Cluster}
    \begin{itemize}
      \item Data stored on a heap is not clustered
    \end{itemize}
  \end{itemize}

\section{Catalogs}

  \begin{figure}
    \centering
    \caption{Attribute\_Cat Table}
    \begin{tabu} to \columnwidth{| X[1, l] | X[1, l] | X[1, l] | X[1, l] |}
      \hline
      \bfseries{attr\_name} & \bfseries{rel\_name} & \bfseries{type} & \bfseries{position} \\ \hline \hline
      attr\_name & Attribute\_Cat & string & 1 \\ \hline
      rel\_name & Attribute\_Cat & string & 2 \\ \hline
      type & Attribute\_Cat & string & 2 \\ \hline
      sid & Students & string & 1 \\ \hline
      name & Students & string & 2 \\ \hline
    \end{tabu}
  \end{figure}

  \begin{itemize}
    \item \emph{Catalogs are tables}; attributes of the catalog table is also
    stored in the catalog
    \item Catalogs contain at least
    \begin{itemize}
      \item \emph{Number of tuples and pages} for reach \emph{relation}
      \item \emph{Number of distinct key values and pages} for each \emph{index}
      \item \emph{Index height, low and high key values} for each tree \emph{index}
    \end{itemize}

    \item Catalogs are updated periodically
  \end{itemize}

\section{Access Paths}

  \begin{itemize}
    \item An \emph{access path is a method of retrieving tuples}: file scan or
    index that matches a selection (in the query)
    \item A tree index matches (a conjunction of) terms \emph{involve only attributes
    in a prefix of the search key}
    \begin{itemize}
      \item Ex. a tree index on $ <a, b, c> $ matches the selection
      $ a = 5 \AND b = 3 $, and $ a = 5 \AND b > 6 $, but
      not $ b = 3 $; b is not in the prefix;
      \item $ a = 5 \AND b = 3 $ is \emph{the search key}
    \end{itemize}

    \item A hash index matches (a conjunction of) terms that \emph{has a term
    $ attribute = value $ for every attribute in the search key} of the index
    \begin{itemize}
      \item Ex. hash index on $ <a, b, c> $ matches the
      \begin{itemize}
        \item $ a = 5 \AND b = 3 \AND c = 5 $
      \end{itemize}
      but not
      \begin{itemize}
        \item $ b = 3 $: not every attribute is included
        \item $ a = 5 \AND b = 3 $: not every attribute is included
        \item $ a > 5 \AND b = 3 \AND c = 5 $: $ a > 5 $
      \end{itemize}
    \end{itemize}

    \item Selection conditions are first converted to \emph{conjunctive normal form (CNF)};
    CNF: terms connected by $ \AND $
    \begin{itemize}
      \item Every boolean expression can be converted to CNF
    \end{itemize}
  \end{itemize}

\section{Selection}

  \begin{itemize}
    \item Find \emph{the most selective access path}, retrieve tuples using it and apply
    any remaining terms that don't match the index
    \item \emph{The most selective access path}: an index or file scan that
    we estimate would require the fewest page $ I/Os $
    \item \emph{Terms that match the index reduce} the number of \emph{tuples retrieved};
    \emph{other terms} are used to \emph{discard retrieved tuples}, \emph{but do not
    affect the number of tuples/pages fetched};
    \begin{itemize}
      \item Given $ day < 8/9/94 \AND bid = 5 \AND sid = 3 $, a B+ tree index
      on day would be used and $ bid = 5 \AND sid = 3 $ must be used against
      each retrieved tuple; Or a hash index on $ <bid, sid> $ can be used
      and $ day < 8/9/94 $ can be used
    \end{itemize}
  \end{itemize}

  \subsection{Using an Index for Selections}

    \begin{itemize}
      \item Cost depends on
      \begin{align*}
        \#\text{ qualifying tuples and clustering }
          &= \#\text{ qualifying data entries } \\
          &+ \text{ cost of retrieving actual records }
      \end{align*}
    \end{itemize}

\section{Projection}

  \begin{lstlisting}
SELECT DISTINCT R.sid, R.bid FROM Reserves R
  \end{lstlisting}

  \begin{itemize}
    \item Expensive part is removing duplicates
    \item \emph{Sorting approach}: sort on $ <sid, bid> $ and remove duplicates
    \begin{itemize}
      \item If there is an index on both $ sid, bid $, cheaper to sort data
      entries
    \end{itemize}
    \item \emph{Hasing approach}: hash on $ <sid, bid> $ to create partitions.
    Load partitions one at a time and create in memory hash structure
  \end{itemize}

\section{Join}

  \subsection{Block Nested Loop}

    \begin{equation*}
      R \Join_{i = j} S
    \end{equation*}

    \begin{lstlisting}
foreach tuple r in R do
  foreach tuple s in S where ri == sj do
    add <r, s> to result
    \end{lstlisting}

    \begin{itemize}
      \item If there is an index on the joined column of one relation, make it inner
      and exploit the index
      \begin{itemize}
        \item \emph{Cost}: $ M + \left( \left( M * p_{R} \right) * \text{ cost of finding matching S tuples} \right) $
        \item $ M = \# \text{ pages of R } $
        \item $ p_{R} = \# \text{ R tuples per page} $
      \end{itemize}

      \item For each $ R $ tuple, cost of probing S is
      \begin{itemize}
        \item 1 for hash index
        \item 3 for B+ tree
        \item Clustered index: 1 IO typical, unclustered: upto 1 IO per matching
        S tuple
      \end{itemize}

      \item $ R \Join S, S \Join R $ gives the same result, at different performance
      \begin{itemize}
        \item Better to have smaller table on the left of $ \Join $
        \item Performed by query optimizer
      \end{itemize}
    \end{itemize}

  \subsection{Sort Merge}

    \begin{align}
      cost &= N \log N + M \log M + \left( N + M \right) \\
      cost &= 3 N + 3 M + \left( N + M \right)
    \end{align}

    \begin{itemize}
      \item Cost of \emph{scanning (aka. merging)} $ N + M $, could be $ N \times M $
      \item In practice, the \emph{cost of sorting} can be $ 3 N $
      \item Sort $ R, S $ on the join column, then scan them to do a merge
      and output the result tuples
      \item $ R $ is scanned once, each $ S $ group is scanned per matching
      $ R $ tuple. (Multiple scans of an S group are likely to find needed pages in buffer.)
      \item Preferred algorithm when tree or hash index is already available
    \end{itemize}

\section{System R Optimizer}

  \begin{itemize}
    \item Most widely used
  \end{itemize}

  \subsection{Cost Estimation}

    \begin{lstlisting}
SELECT attribute list FROM relation list WHERE term1 AND ... AND termk;
    \end{lstlisting}

    \begin{align}
      \text{result cardinality } &= \max \# tuples * \text{ product of all RFs}
    \end{align}

    \begin{itemize}
      \item \emph{Estimate the cost of each operation} in plan tree
      \item \emph{Estimate size of result} for each operation in plan tree
      \item \emph{Reduction factor (RF)}: associated with each term reflects
      the impact of the term in reducing result size
      \begin{itemize}
        \item Assume terms are independent
        \item $ col = value $ has RF $ 1 / \text{ number of possible values} $
        \item $ col > value $ has RF
        $ \left( \high(I) - value \right) / \left( \high(I) - \low(I)  \right) $
      \end{itemize}
    \end{itemize}
