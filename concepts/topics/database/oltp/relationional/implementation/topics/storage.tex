\chapter{Storage}

\begin{itemize}
  \item \emph{Disk Block}: data on disk is read and written in disk blocks
  \item \emph{Page}: made of multiple blocks
  \item \emph{READ}: fetch data from disk to main memory
  \item \emph{WRITE}: flush data from main memory to disk
  \item \emph{Time to access a disk block}: $ seek + rotation + transfer $
  \begin{itemize}
    \item Seek and rotation dominiate
  \end{itemize}
\end{itemize}

\section{Disk Space Management}

  Lowest layer of DMBS manages disk space

  \begin{enumerate}
    \item \code{allocate(int n)}: allocate n pages, returns the pid of the
    first page of the sequence of n pages
    \item \code{deallocate(int pid, int n)}: free n consecutive pages starting
    from pid.
  \end{enumerate}

\section{Buffer Management}

  \begin{enumerate}
    \item If requested page is not in pool
    \begin{enumerate}
      \item Choose a frame to be replaced
      \item If the frame is dirty, write it to disk
      \item Read request page into chosen frame
    \end{enumerate}

    \item Pin the page and returns its address
    \item If requests can be predicted, pages can be \emph{pre-fetched}
  \end{enumerate}

  \subsection{DBMS vs OS File System}

    \begin{itemize}
      \item Portability
      \item OS files cannot span disks
      \item DMBS needs to
      \begin{itemize}
        \item Pin page in a buffer pool, force a page to disk
        \item Adjust replacement policy, and pre-fetch pages
      \end{itemize}
    \end{itemize}

\section{Page Formats}

  \begin{equation}
    R = (p, n)
  \end{equation}

  \begin{itemize}
    \item Record ID composes of page id $ p $, and the index of the record in
    page $ n $
    \item Page composes of
    \begin{itemize}
      \item \emph{Pointer to start of free space}
      \item A number $ N $, \emph{number of slots}
      \item \emph{Slot directory}: an array where element is the pointer to
      a record and index of the element is $ n $
      \item \emph{Records}
    \end{itemize}
  \end{itemize}

\section{Files}

  \begin{itemize}
    \item Higher levels of DBMS operate on records and files of records
    \item \emph{File}: a collection of pages, each containing a collection
    of records. Must support
    \begin{itemize}
      \item Insert, delete and modify command
      \item Read a particular record using record id
      \item Scan all records
    \end{itemize}
  \end{itemize}

  \subsection{Heap}

    \begin{itemize}
      \item Contain records with no particular order
      \item As files grow and shrink, disk space are allocated and deallocated
    \end{itemize}

    \subsubsection{Poissible Implementations}

      \begin{itemize}
        \item \emph{List}: a header page contains link to two doubly linked list,
        one for full pages and one for pages with free space
        \item \emph{Page directory}: header pages contains records to data pages
        and how many space are available on these data pages
      \end{itemize}

\section{Index}

  \begin{itemize}
    \item An \emph{index} on a \emph{file} speeds up selectino on the search key fields
    for the index
    \begin{itemize}
      \item Any subset of the fields of a relation can be the search key for
      an index on the relation
      \item Search key is not the same as key\footnote{minimal set of
      fields that uniquely identify a record in a relation}
    \end{itemize}

    \item A index contains a collection of data entries.
    \item A index supports efficient retrieval of all data entries $ k* $
    with the given key $ k $
  \end{itemize}

  \subsection{Hash}

    \begin{itemize}
      \item Good for equality selections
      \item Index is a collection of \emph{buckets}
      \item A \emph{bucket} is \emph{primary page plus zero or more overflow pages}
      \item \emph{Hashing function}:
      \begin{itemize}
        \item $ h(r) $ bucket in which record $ r $ belongs.
        $ h $ looks at the search key fields on $ r $
        \item $ N = \#\text{ of buckets}, h(k) \% N = $ buckets to which data
        entry with key $ k $ belongs
      \end{itemize}
    \end{itemize}

  \subsection{BTree}

    \begin{itemize}
      \item \emph{Leaf} page \emph{contain data entries} and \emph{are chained}
      \item \emph{Non-leaf} pages contain \emph{index entries}, facilitating navigation
      to lower level node and eventually the leaf node
      \item Tree is always height balanced
      \item \emph{Nodes are at least 50\% full (except for root)}
      \item Supports both \emph{equality and range search}
    \end{itemize}

  \subsection{Primary vs Secondary Index}

    Primary vs secondary index is \emph{orthogonal to the indexing used}

    \begin{itemize}
      \item \emph{Primary index}: the data entry k* in the index contains
      the data record with search key value k
      \begin{itemize}
        \item \emph{Pro}: fast, avoids an extra IO access
        \item \emph{Con}: at most one primary index on a given collection of
        data records (otherwise, records are duplicated)
        \item \emph{Con}: large data entries, more pages for data, size of
        index is too large
      \end{itemize}

      \item \emph{Secondary index}: the data entry k* in the index is of the form
      \begin{equation}
        <k, \text{RID of data record with search key value k}>
      \end{equation}
      \begin{itemize}
        \item \emph{Pro}: any number of secondary indices
        \item \emph{Pro}: data entry smaller than data records, fewer leaf pages,
        smaller index
        \item \emph{Con}: an extra IO to fetch data record
      \end{itemize}
    \end{itemize}

  \subsection{Clustering}

    If order of data records is the same as or close to order of data
    entries, then it is called a clustered indx

    \begin{itemize}
      \item Primary index is clustered
    \end{itemize}
