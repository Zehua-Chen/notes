\chapter{Transaction}

\begin{itemize}
  \item DBMS is only concerned with read and writes
  \item A transactino is the DBMS's abstract view of a user program, a series
  of read and writes
  \item Two transactions $ T1, T2 $ submitted in order:
  \begin{enumerate}
    \item No guarantee that $ T1 $ run before $ T2 $
    \item Result should be as if $ T1 $ run before $ T2 $
  \end{enumerate}
\end{itemize}

\section{Concurrency}

  \begin{itemize}
    \item DBMS interleave read and writes of various transactions
    \item Each transaction leave the database in a consistent state if DB is
    consistent when the transaction begins
  \end{itemize}

  \subsection{Atomicity}

    \begin{itemize}
      \item \emph{Commit}: a transaction becomes permanent in the system
      \item \emph{Abort}: a transaction does not become permanent in the system
      \item \emph{Atomicity}: user can think of acts in a transaction as always
      executing its actions in one step or not executing at all
      \begin{itemize}
        \item A log is kept for redo
      \end{itemize}
    \end{itemize}

  \subsection{ACID Properties}

    \begin{itemize}
      \item \emph{Atomicity}: all actions happen or none happen
      \item \emph{Consistency}: if each Xact is consistent and DB starts consistent,
      then DB ends consistent
      \item \emph{Isolation}: execution of one Xact is isolated from that of
      other Xacts
      \item \emph{Durability}: if a Xact commits, its effects persist
    \end{itemize}

    \emph{Concurrency Control Manager} guarantees C, I; \emph{Recovery manager}
    guarantees A, D

\section{Schedules}

  \begin{itemize}
    \item A \emph{concurrent (interleaved) execution} of two or more
    transactions is \emph{correct if their execution is serializable}
    \begin{itemize}
      \item Ex. Given $ T1, T2 $. So long as the result is the same as if $ T1 \to T2 $,
      or $ T2 \to T1 $, then the the result is correct
    \end{itemize}
    \item \emph{Equivalent schedule} the result of executing two schedules
    are the same
  \end{itemize}

  \subsection{Serial Schedule}

    \emph{Serial schedule}: Schedule that does not interleave the actions of
    different transactions

    \begin{itemize}
      \item Correct by definition
    \end{itemize}

  \subsection{Serializable Schedule}

    \emph{Serializable schedule}: a schedule that is equivalent to \emph{some
    serial execution of the transactions}

\section{Anomolies with Interleaved Operations}

  \begin{itemize}
    \item \emph{Reading uncommited (aborted) data}: WR conflicts, dirty reads;
    transaction 1 abort write operation, transaction 2 read aborted, i.e. dirty data
    \item \emph{Unrepeatable reads}: RW conflicts; one transaction reads twice,
    expecting the same value, but a write is interleaved, causing the second read
    to have different value
  \end{itemize}

  \subsection{Conflicting Operations}

    Two operations conflict if the order in which they are performed matters

    \begin{itemize}
      \item $ \R\left( x \right), \W\left( x \right) $
      \begin{itemize}
        \item $ \R $ conflicts with $ \W $
        \item $ \W $ conflicts with both $ \R, \W $
      \end{itemize}

      \item \emph{Commits and aborts are not conflicting operations}
    \end{itemize}

  \subsection{Test for Serializability}

    \begin{itemize}
      \item Build the precedence graph (PG) of schedule S
      \item PG is a \emph{directed graph}
      \begin{itemize}
        \item Nodes represents transactions in S
        \item Edge between $ T_{i}, T_{j} $ means that one of $ T_{i} $'s
        operation precedes and conflicts one of $ T_{j} $'s operations
      \end{itemize}

      \item A schedule \emph{is serializable if its PG has no cycles}
    \end{itemize}

    \subsubsection{Building PG}

      \begin{itemize}
        \item $ \R_{i} (x), ..., \W_{j} (x), i \ne j $: edge from $ i $ to $ j $
        \item $ \W_{i} (x), ..., \W_{j} (x), i \ne j $: edge from $ i $ to $ j $
        \item $ \W_{i} (x), ..., \R_{j} (x), i \ne j $: edge from $ i $ to $ j $
      \end{itemize}

  \subsection{Finding Equivalent Serial Schedule}

    \begin{itemize}
      \item \emph{If there is no cycle}, make a topological sorting of PG
      \item \emph{Algorithm}: start with an empty sorted list and repeat
      until you can no longer do this
      \begin{itemize}
        \item Find a node N with no incoming edges, remove N, and all N's
        outgoing edges from the graph. Append N to the sorted list
      \end{itemize}

      \item Several linear orders can be obtained
    \end{itemize}

  \subsection{Lock-Based Concurrency Control}

    \begin{itemize}
      \item \emph{Strict Two-phase Locking (Strict 2PL) Protocol}:
      \begin{itemize}
        \item Each Xact must obtain a S(shared) lock on object before reading,
        and an X(exclusive) lock on object before writing
        \item All locks held by a transaction are released when the transaction
        completes
        \begin{itemize}
          \item Non strict 2PL: release lock anytime, but cannot acquiare locks
          after releasing any locks
        \end{itemize}

        \item If an Xact holds an X lock on an object, no other Xact
        can get a lock (S or X) on that object
        \item If an Xact holds a S lock on an object, no other Xact can get a
        X lock on that object
      \end{itemize}

      \item \emph{Allows only serializable schedules}
      \begin{itemize}
        \item Simplifies transaction aborts
        \item Non-strict 2PL \emph{also} allows only serializable schedules,
        but involes \emph{more complex abort processing}
      \end{itemize}

      \item DB lock guarantee the correctness of entire set of operations, i.e.
      transactions, OS lock guarantee the correctness of a single operation
    \end{itemize}

    \subsubsection{Deadlocks}

      \begin{itemize}
        \item Various mechanism exists (timeout, WFG)
        \item Scheduler must abort some transaction to resolve the deadlock,
        and the aborted transaction must be restarted
      \end{itemize}

    \subsection{Recovery}
