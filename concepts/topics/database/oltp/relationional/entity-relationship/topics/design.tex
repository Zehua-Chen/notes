\chapter{Database Design}

\begin{itemize}
  \item \emph{Conceptual design} (entity relationship model is used at this
  stage)
  \begin{itemize}
    \item Identify entites and relationships and what information to store about
    them should we store
    \item Identify integrity constraints or business rules
    \item A schema in the ER model can be represented pictorially
    (ER diagrams)
    \item Can map ER diagram into relational schema
  \end{itemize}
  \item \emph{Physical database design and turning}
\end{itemize}

\section{Entity and Relationships}

  \begin{figure}
    \centering
    \begin{tikzpicture}
      \node[circle, draw] (attribute) {Attribute};
      \node[rectangle, below = of attribute, draw] (entity) {Entity};
      \node[diamond, below = of entity, draw] (relationship) {Relationship};

      \draw (attribute) -- (entity);
      \draw (entity) -- (relationship);
    \end{tikzpicture}
  \end{figure}

  \begin{itemize}
    \item \emph{Entity} real-world objects distinguishable from other objects;
    described using a set of attributes
    \begin{itemize}
      \item \emph{Attribute}: holds \emph{a single value} of some
      domtain/type (no arrays, structures, subsets, ...)
      \item Represented using rectangles
    \end{itemize}

    \item \emph{Entity set}: a collection of similar entities
    \begin{itemize}
      \item Share the same set of attributes
      \item Each entity set has \emph{a unique key}
    \end{itemize}

    \item \emph{Relationship}: association among entities ($ > 1 $)
    \begin{itemize}
      \item Attributes are optional
      \item Types
      \begin{enumerate}
        \item 1 to 1
        \item 1 to many
        \item Many to 1
        \item Many to many
      \end{enumerate}
    \end{itemize}

    \item \textbf{Relationship set}: a collection of similar entities
    \begin{itemize}
      \item A $ n $-nary relationship set $ R $ relates $ n $ entity set
      $ E_{1}, ..., E_{n} $
      \item Same entity set can participates in different relationship set,
      or in different roles in the same set
    \end{itemize}
  \end{itemize}

\section{Constraints}

  \begin{itemize}
    \item \emph{Key and Participation Constraints}
    \begin{itemize}
      \item Many to 1
      \item Many to many
    \end{itemize}

    \item \emph{Key constraints}
    \item \emph{Participation constraints}
  \end{itemize}
