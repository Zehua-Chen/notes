\chapter{Software}

\section{Virtual World: Real vs Synthetic}

  \begin{itemize}
    \item The virtual world can be either real or synthetic
    \begin{itemize}
      \item \textbf{SLAM}: Simultaneous Localization and Mapping
    \end{itemize}
  \end{itemize}

\section{Matched Motion}

  \begin{itemize}
    \item \textbf{Matched zone}: the safe zone the user is confined in
    \begin{itemize}
      \item Challenge: sync obstacles between real and virutal worlds
      \item Larger matched zones have safety issues
    \end{itemize}

    \item The renderer has to use the right viewport into hte virutal
    world to create an immersive experience
  \end{itemize}

\section{User Locomotion}

  \begin{itemize}
    \item Moving outside of matched zone motivates \textbf{locomotion}
    \item Lead to sickness
  \end{itemize}

\section{Physics}

  \begin{itemize}
    \item Physics make the virtual world behave like real world
    \begin{itemize}
      \item Collision detection
    \end{itemize}

    \item Also handle how potential stimuli (ex. light) are created and
    propagate through the virtual world
  \end{itemize}

\section{Networked Experiences}

  \begin{itemize}
    \item The server maintains a shared VWG
  \end{itemize}

\section{Rendering}

  \begin{itemize}
    \item Lots of pressure ot increase framerate (reduce complexity)
    \item Heuristics that worked well for screen-based graphics may be
    perceptily wrong in VR
  \end{itemize}
