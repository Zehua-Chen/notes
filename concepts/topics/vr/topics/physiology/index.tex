\chapter{Physiology and Perception}

\section{Overview}

  \begin{description}
    \item[Proprioception] the ability to sense the relative positions
    of parts of our bodies and the amount of muscular effort being
    involved in moving them.
    \item[Adaptation] The perceived effect of stimuli changes over time
    \begin{itemize}
      \item FPS players experience less vection in VR
    \end{itemize}

    \item[Just Noticeable Difference (JND)] This is the amount that the
    stimulus needs to be changed so that subjects
    would perceive it to have changed in at least 50 percent of trials.
  \end{description}

  \subsection{Stevens's Power Law}

    \textbf{Stevens's power law} characterizes the \textbf{exponential}
    relationship between the increase of intensity of stimulus and the
    perceived magnitude increase in sensation created by the stimulus

  \subsection{Comfort}

    \begin{itemize}
      \item Trials are essential
      \item Need consistent \textbf{90 FPS}
    \end{itemize}

  \subsection{Eyes}

    \subsubsection{Eye Movement}

      \paragraph{Saccades}
      \begin{itemize}
        \item \textbf{Rapid movement}
        \begin{itemize}
          \item 45ms
          \item Rotatino covers 900 degrees per second
        \end{itemize}

        \item Quickly relocates fovea to important features in scene
        \item Can be consciously controlled
      \end{itemize}

    \subsubsection{Vergence Accommodation Mismatch}

      \begin{itemize}
        \item Eyes are accommodating to the screen distance
        \item Eyes converge to some virtual depth
        \item Active area of research
      \end{itemize}

    \subsubsection{Vestibulo Occular Reflex (VOR)}

      \begin{itemize}
        \item Involuntary movement of the eye to counter head movement
        \item Provides image stablization
      \end{itemize}

    \subsubsection{Others}

      \begin{itemize}
        \item Smooth Pursuit
        \item Optokinetic
        \item Vergence
        \item Microsaccadess
      \end{itemize}

    \subsection{Implications for VR}

      \begin{itemize}
        \item Intensity resolution
        \item Temporal resolution
      \end{itemize}

      \paragraph{Sptial resolution}
      \begin{itemize}
        \item \textbf{Retina display}: 326 pixels per linear inch
        \item Low resolutions causes:
        \begin{itemize}
          \item Aliasing
          \item Screen door effect
        \end{itemize}

        \item \textbf{Cycle per degree}: number of stripes that can be seen
        as separate along a viewing arc of 1 degree
        \item VR needs 2291.6 PPI to achieve 30 cycles per degree
        \item \textbf{Field of view} maximum human field of view is around
        270 degrees; how does a human achieve that FOV without head movement?
        \item \textbf{Foveated Rendering} draw the region of the scene the fovea
        is looking at a higher resolution; draw everything else on a lower
        resolution
        \item \textbf{VOR Gain} is a ratio between the eye rotation rate and
        the rotation and translate rate of the head
        \begin{itemize}
          \item Brain will adapt VOR to different optics (happen to people with
          glasses)
          \item Also happens with bad VR displays (increases happiness with VR
          experience; requires eyeadjustments outside VR)
        \end{itemize}

        \item \textbf{Display Scanout}
        \begin{itemize}
          \item Rolling scanout: updating pixels line by line
          \item For 60FPS, could take up to 16.67 ms
        \end{itemize}

        % TODO: Rewatch last part of lecture
      \end{itemize}

\subimport{./}{depth}
\subimport{./}{motion}
