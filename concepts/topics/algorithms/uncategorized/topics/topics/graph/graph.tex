\chapter{Graph}

\section{Undirected Graph}

  Algorithmic Problems

  \begin{itemize}
    \item Given graph $ G $, and node $ u, v $, is $ u $ connected to $ v $?
    \begin{itemize}
      \item Can be solved using BFS and DFS
    \end{itemize}

    \item Given $ G $ and node $ u $, find all nodes that are connected to $ u $
    \begin{itemize}
      \item Can be solved using BFS and DFS
    \end{itemize}

    \item Find all connected components of $ G $
    \begin{itemize}
      \item Can be solved using BFS and DFS
    \end{itemize}
  \end{itemize}

  \subsection{Traversal}

    Graph traversal typically requires two datastructures

    \begin{enumerate}
      \item Visisted list
      \item \say{To explore} list
      \begin{itemize}
        \item BFS uses a queue
        \item DFS uses a stack
      \end{itemize}
    \end{enumerate}

    Both BFS and DFS can be done in $ O\left( n + m \right) $ time

  \subsection{DFS}

    \subsubsection{Tree}

      \begin{itemize}
        \item $ T $ is a forst
        \item Root of the tree is the first vertex iterated upon
        \item Nodes are in the tree iff they are reachable from the root vertex
        in the original graph
        \item Edges can be classified into two types
        \begin{itemize}
          \item \textbf{Tree edge}: belongs to the tree
          \item \textbf{Non tree edge}: does not belogn to the tree
          \begin{itemize}
            \item \textbf{Forward edge}:
            \begin{align*}
              \left( x, y \right):
              \pre\left( x \right)
              < \pre\left( y \right)
              < \post\left( y \right)
              < \post\left( x \right)
            \end{align*}
            Or in aonther word, go down the tree

            \item \textbf{Backward edge}:
            \begin{align*}
              \left( y, x \right):
              \pre\left( x \right)
              < \pre\left( y \right)
              < \post\left( y \right)
              < \post\left( x \right)
            \end{align*}
            Or in aonther word, go up the tree

            \item \textbf{Cross edge}: the intervals of two vertices in the
            edge are disjoint; no child or parent relationship
          \end{itemize}
        \end{itemize}

        \item If $ uv $ is a non tree edge, then in $ T $, one of the following
        conditions can be true
        \begin{itemize}
          \item $ u $ is an ancestor of $ v $
          \item $ v $ is an ancestor of $ u $
        \end{itemize}
      \end{itemize}

    \subsubsection{Pre, Post Numbers}

      \begin{itemize}
        \item Node $ u $ is active in the interval
        \begin{align*}
          \left[ \pre\left( u \right), \post\left( u \right) \right]
        \end{align*}

        \item For two nodes $ u $ and $ v $, either there intervals are
        \textbf{disjoint} or one is contained in the other
      \end{itemize}

  \subsection{BFS}

    \begin{itemize}
      \item BFS uses a queue to store the list of vertices to explore
      \item BFS would always visists the vertices that are closer to the
      starting vertices first
    \end{itemize}

    \subsubsection{Theorems}

      The following holds true upon completion of $ \BFS\left( s \right) $ in
      \textbf{undirected graphs}

      \begin{enumerate}
        \item The search tree contains exactly the vertices in the connected
        component of $ s $
        \item If $ \dist\left( u \right) < \dist\left( v \right) $, then $ u $
        is visisted before $ v $
        \item For every $ u $, $ \dist\left( u \right) $ is the length of the
        shortest path (number of edges from $ s $ to $ u $)
        \item If $ u, v $ are in the connected component of $ s $, and
        $ e = \left\{ u,v \right\} $ is an edge of $ G $ then
        \begin{equation}
          \left| \dist\left( u \right) - \dist\left( v \right) \right| \le 1
        \end{equation}
      \end{enumerate}

  \subsection{BFS with Layers}

    $ \BFSLayers\left( s \right) $ instead of putting to explore nodes into a
    queue, put them into layers

    \begin{itemize}
      \item \textbf{Runtime}: $ O\left( n + m \right) $
      \item Outputs a BFS tree
      \item $ L_{i} $ is the set of vertices at a shortest distance exactly
      $ i $ from $ s $
    \end{itemize}

    \subsubsection{Theorems}

      If $ G $ is undirected, then each edge $ e = \left\{ u, v \right\} $
      is

      \begin{enumerate}
        \item Tree edge between two consecutive layers
        \item Non-tree forward, backward edge between two consecutive layers
        \item Non-tree cross edge with $ u, v $ in the same layer
        \item Every edge in the graph between two vertices is either of the
        followings
        \begin{enumerate}
          \item In the same layer
          \item In two consecutive layers
        \end{enumerate}
      \end{enumerate}

\section{Directed Graph}

  \begin{itemize}
    \item Given $ G $ and nodes $ u, v $, can $ u $ reach $ v $?
    \begin{itemize}
      \item $ O\left( n + m \right) $
    \end{itemize}
    \item Given $ G $ and node $ u $, compute $ \rch\left( u \right) $?
    \begin{itemize}
      \item $ O\left( n + m \right) $
    \end{itemize}

    \item Given $ G $ and node $ u $, compute all $ v $ that can reach $ u $,
    $ u \in \rch\left( u \right) $?
    \begin{itemize}
      \item \textbf{Naive}: $ O\left( n \left( n + m \right) \right) $
      \item \textbf{Smart}: compute $ G^{rev} $ ($ O\left( n + m \right) $),
      compute $ \rch\left( u \right) $ via search ($ O\left( n + m \right) $);
      if $ \Out\left( v \right), \In\left( v \right) $ are all available at
      each $ v $, then there is no need for $ G^{rev} $
      \begin{itemize}
        \item Instead of using $ \Out\left( v \right) = \In\left( v \right) $,
        use $ \ln\left( v \right) $
      \end{itemize}
    \end{itemize}

    \item $ \SCC\left( u \right) $
    \begin{itemize}
      \item $ \SCC\left( G, u \right) = \rch\left( G, u \right) \rch \left( G^{rev}, u \right) $
      ($ O\left( n + m \right) $) time
    \end{itemize}

    \item Is $ G $ strongly connected
    \begin{itemize}
      \item Pick arbitrary vertex $ u $, check if
      $ \SCC\left( G, u \right) = V $
    \end{itemize}

    \item Find all strongly connected components of $ G $
    \begin{itemize}
      \item \textbf{Brue force}: while $ G $ is not empty, pick an arbitrary
      vertex $ u $, remove $ \SCC\left( G,u \right) $ from $ G $
      ($ O\left( n \left( n + m \right) \right) $ time)
      \item \textbf{Linear Time}: Can be done in $ O\left( n + m \right) $ time
      \ref{subsec: linear time sccs}
    \end{itemize}
  \end{itemize}

  \subsection{Determining DAG}

    \begin{enumerate}
      \item Compute DFS
      \item Not DAG if there is an backward edge $ e = \left( v, u \right) $
      Output cycle formed by path from $ u $ to $ v $ in $ T $ plus edge $ vu $
      \item Otherwise output nodes in decreasing post-visit order.
      \item $ O\left( n + m \right) $ time
    \end{enumerate}

  \subsection{Linear Time Finding SCCs} \label{subsec: linear time sccs}

    \paragraph{Outline}
    \begin{enumerate}
      \item Do DFS on the reverse graph and output intervals in decreasing post
      order
      \item Mark all vertices as unvisisted
      \item For each vertex in the computed order
      \begin{itemize}
        \item if the vertex is not visisted
        \begin{itemize}
          \item DFS the vertex
          \item Output the vertices reached by the vertex
          \item Remove the outputed vertices from the graph
        \end{itemize}
      \end{itemize}
    \end{enumerate}

    \paragraph{Running Time}
    \begin{itemize}
      \item DFS would only visit vertices SCCs
      \item DFS takes proportional time to the size of the SCC
      \item Therefore running time is $ O\left( n + m \right) $
    \end{itemize}

  \subsection{BFS}

    \subsubsection{Theorems}

      The following hold true upon completion of $ \BFS\left( s \right) $
      in \textbf{directed graphs}

      \begin{enumerate}
        \item Tree contains exactly the set of vertices \textbf{reachable}
        from $ s $
        \item if $ \dist\left( u \right) < \dist\left( v \right) $, then
        $ u $ is visited before $ v $
        \item For every $ u $, $ \dist\left( u \right) $ is the shortest length
        from $ s $ to $ u $
        \item If $ u $ is reachable from $ s $, and
        $ e = \left\{ u, v \right\} $ is an edge of $ G $ then
        \begin{equation}
          \dist\left( v \right) - \dist\left( u \right) \le 1
        \end{equation}
      \end{enumerate}

  \subsection{BFS with Layers}

    If $ G $ is directed, then each edge $ e = \left\{ u, v \right\} $
    is:

    \begin{itemize}
      \item Tree edge between two consecutive layers
      \item A non-tree forward edge between two consecutive layers
      \item A non-tree backward edge
      \item A cross edge with both $ u, v $ in the same layer
    \end{itemize}

\subimport{./}{shortest-path-positive}
\subimport{./}{shortest-path-negative}
