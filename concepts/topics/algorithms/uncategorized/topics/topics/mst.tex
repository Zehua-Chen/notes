\chapter{Minimum Spanning Tree}

\paragraph{Problem} Given $ G $, we want to construct a tree such that
all vertices from $ G $ are connected and we want to minimize the total cost
of edges


\paragraph{Applications}
\begin{itemize}
  \item Network Design
  \item Approximation algorithm
  \item Cluster analysis
\end{itemize}

\paragraph{Solution} Solvable using greedy algorithm with the main problems
\begin{itemize}
  \item In what orders should the edges be processed
  \item When to add edge to the tree
\end{itemize}

\begin{algorithm}[H]
  \Begin{
    \tcp{the MST tree}
    $ T \gets \left\{ \right\} $\;
    \While{$ \left| E \right| > 0 $}{
      \underline{Pick} $ e \in E $\;

      \If{$ e $ \underline{satisfies condition}}{
        Add $ e $ to $ T $\;
      }
    }
  }
\end{algorithm}

\section{Properties of MST Tree}

  \begin{itemize}
    \item A graph is connected iff a MST is present
    \item MST of $ n $ nodes has $ n - 1 $ edges
    \item If we add an edge $ e $ that is not the in the tree to the tree $ T $,
    there would be a unique cycle $ C $
    \begin{itemize}
      \item For $ f \in C - \left\{ e \right\} $, $ T - f + e $ is another
      spanning tree of $ G $
    \end{itemize}
  \end{itemize}

  \subsection{Cuts}

    A cut is a partition of the vertices of the graph into two sets
    $ S, \left( V \setminus S \right) $

    \begin{itemize}
      \item Edges having endpoints on both side are the edges of the cut
      \item A cut edge is crossing the cut
    \end{itemize}

  \subsection{Safe, Unsafe}

    \begin{itemize}
      \item An edge $ e $ is \textbf{safe} if there is a cut such that among all
      \textbf{edges of the cut}, $ e $ is the unique smallest cost
      \begin{itemize}
        \item Given a safe edge, all MST contains this edge
        \item Safe edges form MST
        \item Safe edges form connected graph
      \end{itemize}

      \item An edge $ e $ is \textbf{unsafe} if there is a cycle in which
      $ e $ is the unique maximum cycle
      \begin{itemize}
        \item No MST contains unsafe edges
      \end{itemize}

      \item All edges are distinct if the edges are all either safe or unsafe
    \end{itemize}


\section{Kruskal’s Algorithm}

  Process edges in the order of costs, and add edges to $ T $ so long as they
  don't form cycles.

\section{Prims's Algorithm}

  Start with a node, add edges that induce the smallest attachment cost
  to the tree

\section{Reverse Delete}

  \begin{algorithm}[H]
    $ T \gets E $\;
    \While{$ \left| E \right| > 0 $}{
      Choose $ e \in E $ of the largest cost\;

      \If{removing $ e $ does not disconnect $ T $}{
        remove $ e $ from $ T $\;
      }
    }
  \end{algorithm}

\section{Boruvka's Algorithm}

  Assume $ G $ is a connected graph

  \begin{algorithm}[H]
    $ T \gets \emptyset $\;
    \While{$ T $ is not spanning}{
      $ X \gets \emptyset $\;
      \For{each connected component $ S \in T $}{
        Add to $ X $ the cheapest edge between $ S $ and $ V \setminus S $\;
      }

      Add edges in $ X $ to $ T $\;
    }
  \end{algorithm}


\section{Special Edges}

  \subsection{Duplicate Edge Costs}

    Break ties using index of edges (lexicographic ordering). The following
    algorithms are optimal using lexicographic ordering
    \begin{itemize}
      \item Prims's
      \item Kruskal’s
      \item Reverse Delete Algorithm
    \end{itemize}

  \subsection{Negative Edge Costs}

    MST algorithms work with negative edges without modifications

\section{Implementation and Running Time}

  Prims's algorithm has the best running time

  \subsection{Kruska's Algorithm}

    \begin{itemize}
      \item Presort edges based on cost so that choosing is constant time
      \item Do BFS on $ T \cup \left\{ u \right\} $, takes $ O\left( n \right) $
      \item Total
      \begin{equation}
        O\left( m \log m \right) + O\left( m n \right) = O\left( m n \right)
      \end{equation}

      \item Can give $ O\left( \left( m + n \right) \log m \right) $ if done
      using union find algorithm
    \end{itemize}

    \begin{algorithm}[H]
      \caption{Using Union Find}

      \Begin{
        $ T \gets \emptyset $\;
        Put each vertex $ u $ into its own set\;

        \While{$ \left| E \right| > 0 $}{
          pick $ e = \left( u, v \right) \in E $ of minimum cost\;
          \If{$ u $ and $ v $ belongs to different sets}{
            add $ e $ to $ T $\;
            merge the set containing $ u $ and $ v $\;
          }
        }
      }
    \end{algorithm}

  \subsection{Prims's Algorithm}

    \begin{itemize}
      \item Standard heap takes $ O\left( \left( n + m \right) \log n \right) $
      \item Fibonacci heap takes $ O\left( n \log n + m \right) $
      \item If $ m $ is $ O\left( n \right) $, then
      $ \Omega\left( n \log n \right) $
    \end{itemize}

  \subsection{Boruvka's Algorithm}

    \begin{equation}
      O\left( m \log n \right)
    \end{equation}

    \begin{itemize}
      \item While loop takes $ O\left( \log n \right) $
      \begin{itemize}
        \item Number of connected components shrink by at least half since
        each component merges with one or more other components.
      \end{itemize}
      \item Loop body takes $ O\left( m \right) $
      \item No special data structure needed
    \end{itemize}
