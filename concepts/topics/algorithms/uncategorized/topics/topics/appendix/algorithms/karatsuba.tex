\section{Karatsuba Multiplication}

\subsection{Baseline}

  Given two integers $ x $ and $ y $, they can be written as

  \begin{align*}
    x &= 10^{\frac{n}{2}} x_{l} + x_{r} \\
    y &= 10^{\frac{n}{2}} y_{l} + y_{r}
  \end{align*}

  Therefore $ xy $ can be wrtten as

  \begin{align*}
    xy &=
      \left( 10^{\frac{n}{2}} x_{l} + x_{r} \right)
      \left( 10^{\frac{n}{2}} y_{l} + y_{r} \right) \\
    &=
      10^{n} x_{l} y_{l}
      + 10^{\frac{n}{2}} \left( x_{l} y_{r} + x_{r} y_{l} \right)
      + x_{r} x_{r}
  \end{align*}

  This would require 4 \textbf{recursive} multiplciations;
  multiplying by $ 10^{?} $ can be done using bit-shifts

  \begin{align*}
    T\left( n \right) &= 4 T\left( \frac{n}{2} \right) + O\left( n \right) \\
    &= \Theta\left( n^{2} \right)
  \end{align*}

\subsection{Applying Gauss Trick}

  \begin{equation*}
    x_{l} y_{r} + x_{r} y_{l} =
    \left( x_{l} + x_{r} \right)
    \left( y_{l} + y_{r} \right)
    - x_{l} y_{l} - x_{r} y_{r}
  \end{equation*}

  This means that we only need to recursively compute

  \begin{itemize}
    \item $ \left( x_{l} + x_{r} \right) $
    \item $ \left( y_{l} + y_{r} \right) $
    \item $ x_{l} y_{l} $
    \item $ x_{r} y_{r} $
  \end{itemize}

  The running time would then be

  \begin{align*}
    T\left( n \right)
    &= 3 T\left( \frac{n}{2} \right) + O\left( n \right) \\
    &= O\left( n^{\log_{2} 3} \right) \\
    &= O\left( n^{1.585} \right)
  \end{align*}
