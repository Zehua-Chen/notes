\documentclass[letterpaper, 11pt]{article}
\usepackage{import}
\import{../../../shared/config/}{packages}
\usepackage{graphicx}
\usepackage{float}
\usepackage[ruled, linesnumbered]{algorithm2e}
\usepackage{inconsolata}

\import{../../../shared/config/}{styles}
\import{../../../shared/config/}{macros}

\title{Data Structures - Bounding Volume Hierarchy}
\author{Zehua Chen}

\begin{document}
  \maketitle
  \tableofcontents

  \setmainstyles

  \section{Overview}

    \begin{itemize}
      \item BVH partition elements, not space
      \item Primitives appear at leaf nodes only once;
      \begin{itemize}
        \item In spatial partition, like regular grids, the objects might
        appear in multiple places
      \end{itemize}
      \item Non-leaf nodes only record bounding volume information
    \end{itemize}

    \subsection{Nodes}

      A BVH node is made of

      \begin{itemize}
        \item A left child node
        \item A right child node
        \item A AABB bounding box
      \end{itemize}

      Assuming a binary tree with $ N $ objects:

      \begin{itemize}
        \item $ N $ leaf nodes
        \item $ N - 1 $ interior nodes
      \end{itemize}

    \subsection{Comparison to KD tree}

      \begin{itemize}
        \item BVH is faster to build
        \item BVH is more numerically robust
        \begin{itemize}
          \item Less prone to missed intersections due to round-off errors
        \end{itemize}

        \item KD tree has faster intersection tests
        \item KD tree is a \textbf{spatial partiion}
      \end{itemize}

  \section{Construction}

    \subsection{Median Method}

      \begin{itemize}
        \item Put $ n / 2 $ smallest element to the left
        \item Put $ n / 2 $ biggest element to the right
        \item Done in $ O\left( n \right) $ time
        \begin{itemize}
          \item Median can be found in $ O\left( n \right) $ time
          \item C++: \texttt{std::nth\_element}
        \end{itemize}
      \end{itemize}

    \subsection{Safe Area Method}

      \begin{itemize}
        \item As a \textbf{leaf node}:
        At any time, we make a leaf node of the region that we have;
        any ray will have to test with all objects in the region
        \begin{equation}
          \sum_{i = 1}^{N} t\left( i \right)
        \end{equation}

        $ t\left( i \right) $ is the time to compute a ray intersection with
        the $ i $th primitive; $ N $ is how many primitives are there in the
        node

        \item As an \textbf{interior node}
        \begin{equation}
          c\left( A, B \right) = t_{trav}
            + p_{A} \sum_{i = 1}^{N_{A}} t_{isect} \left( a_{i} \right)
            + p_{B} \sum_{i = 1}^{N_{B}} t_{isect} \left( b_{i} \right)
        \end{equation}

        \begin{itemize}
          \item Assume $ t_{isect} $ is the same for all objects
          \item \textbf{Surface Area Heuristic}: if $ A $ is a parent bounding
          box, and $ B $ is a child bounding box, then the probability of
          a ray hits $ B $, given that it has hit $ A $ is
          \begin{equation}
            P\left( B | A \right) = \frac{s_{B}}{s_{A}}
          \end{equation}
        \end{itemize}

        \item Select the best option
      \end{itemize}

\end{document}
