\chapter{Small World and Big World}

  Social network follows a structure with random perturbation.

  \begin{itemize}
    \item \emph{Small world}:
    \begin{itemize}
      \item Each of us maintains a set of acquaintances
      \item Fast expansion implies that we reach anyone through a few hops
      \item We are all pals
    \end{itemize}

    \item \emph{Big-World}:
    \begin{itemize}
      \item Friends are a graph with limiting expansion
      \item We are biased, factors increase in distance
    \end{itemize}
  \end{itemize}

\section{Arguments Arond Small World}

  \subsection{Skeptics}

    \begin{itemize}
      \item Small world assumes acquaintance set are disjoint
      \item Social acquaintances are \emph{biased against geography, occupation
      and social status} and \emph{favors clusters, inbreeding}
    \end{itemize}

\section{Experiments}

  \subsection{Milgram's Experiment}

    \begin{itemize}
      \item Edges\footnote{edges are connections} are formed
      \begin{itemize}
        \item \emph{Locally}: gender, family/friends
        \item \emph{Globally}: hometown/job
      \end{itemize}

      \item Some people have multiple edges
      \item \emph{Short chains connect us (small world)}
    \end{itemize}

  \subsection{Erdos Number}

    \begin{itemize}
      \item 94\% within distance 6
      \item Median 5
      \item Small number prestigious: field medalist: 2-5, abel prize: 2-4
    \end{itemize}

  \subsection{Kevin Bacon Number}

    \begin{itemize}
      \item Connection is cocllaboration
      \item 88\% actors connected
      \item 98.3\% actors: 4 and less; maximum number is 8
    \end{itemize}

\section{Triadic Closure}

  \begin{definition}
    \emph{\Gls{triadic-closure}}: \glsdesc{triadic-closure}
  \end{definition}

  \begin{itemize}
    \item \emph{Mark Grannovetter's Hypothesis: strong ties should follow triadic
    closure property}
    \item Edge $ e = \left( A, B \right) $ is a local bridge if $ A, B $ have
    no friends in common; \emph{one of these holds}
    \begin{itemize}
      \item B is A's unique strong ties, and A is B's unique strong ties
      \item A or B violates the strong triadic closure
      \item B and A form a weak tie
    \end{itemize}
  \end{itemize}

\section{Balance}

  \begin{definition}
    \emph{\Gls{structural-balance}}: \glsdesc{structural-balance}. Sometimes
    refered to as strong structural balance
  \end{definition}

  \begin{theorem}
    A complete graph is balanced iff it can be divided into \emph{two sets, where
    nodes in the set are mutual friends}, with complete
    \emph{mutual antagonisms between the set}
  \end{theorem}

  \subsection{Incomplete Graph}

    \begin{definition}
      \emph{\Gls{structural-balance-incomplete-graph}}:
      \glsdesc{structural-balance-incomplete-graph}
    \end{definition}
