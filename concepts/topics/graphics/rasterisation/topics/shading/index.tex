\chapter{Shading}

\section{Shading Methods}

  \begin{itemize}
    \item \textbf{Lambertian Shading}
    \item \textbf{Gouraud Shading}
    \begin{itemize}
      \item \textbf{aka. per vertex shading}
      \item Find average normal at each vector
      \item Apply some reflection model at each vertex
      \item Interpolate vertex shades across each polygon
    \end{itemize}

    \item \textbf{Phong Shading}
    \begin{itemize}
      \item \textbf{aka. per pixel shading}
      \item Find average normal at each vertex
      \item Interpoalte vertex normal across edges
      \item Interpolate edge normals across polygon
      \item Apply some reflection model at each fragment
      \item \textbf{Better effect}
    \end{itemize}

    \item Bidirectional Reflectance Distribution Function \textbf{(BRDF)}
    \begin{itemize}
      \item Account for shading in a more general and precise way
    \end{itemize}

    \item \textbf{Global Illumination}
    \begin{itemize}
      \item There is typically a limit on how many reflections are calculated
      when calculating light that bounces between objects
      \item To calculate the bounced light, Lambertian and Blinn-Phong models
      have been used, BDRF are also common
      \item Increasing level of realism come at the cost of increasing
      computation time
    \end{itemize}
  \end{itemize}

\subimport{./}{normals}
\subimport{./}{reflection}
\subimport{./}{rasterization}
