\section{Stage - Projection}

Projection refers to the process of reducing dimensionality. There are two
modes of projection

\begin{itemize}
  \item \textbf{Ortho}: transform view volume
  \begin{itemize}
    \item The orthograhpic matrix only zero-out the z-axis; the ortho matrix
    transform view volume;
    \item Does not actually perform any projection
    \item The actual orthograhpic projection is left for grahpics APIs
  \end{itemize}

  \item \textbf{Perspective}: scale decreases as distance increases
  \item At this stage, everything is already in the context of the camera;
  the view coordinates
  \item \textbf{Foreshortening}: an illusion that cause an object or distance
  to appear shorter than it actually is
  \begin{itemize}
    \item Occur in both orthographic and perspective projection
  \end{itemize}

  \item When \textbf{window resizes}, adjust the projection matrix so that the
  models don't get stretched. For example, instead of using $ w / h $ as the
  aspect, use $ \frac{1}{ w / h} $ as the aspect
\end{itemize}

\subsection{Ortho Projection}

  \paragraph{Matrix}
  \begin{align}
    P_{orthographic} &=
    \begin{bmatrix}
      1 & 0 & 0 & 0 \\
      0 & 1 & 0 & 0 \\
      0 & 0 & 0 & 0 \\
      0 & 0 & 0 & 1
    \end{bmatrix} \\
    P_{ortho} &=
    \begin{bmatrix}
      \frac{2}{r - l} & 0 & 0 & -\frac{r + l}{r - l} \\
      0 & \frac{2}{t - b} & 0 & -\frac{t + b}{t - b} \\
      0 & 0 & \frac{-2}{f - n} & -\frac{f + n}{f - n} \\
      0 & 0 & 0 & 1 \\
    \end{bmatrix}
  \end{align}

  This is a combination of a translation (\say{near} might not be at 0)
  and scale matrix

  \begin{itemize}
    \item $ r $: right
    \item $ l $: left
    \item $ t $: top
    \item $ b $: bottom
    \item $ f $: far
    \item $ n $: near
    \item Some WebGL libraries assumes right-handed system cooridnates as
    inputs, despite itself being left-handed, so its ortho matrix also
    inverses the coordinates along z-axis
  \end{itemize}

\subsection{Perspective Projection}

  \subsubsection{Matrix}

    DirectX, Metal (Asymmetrical)

    \begin{align*}
      \begin{bmatrix}
        \frac{2 z_{near}}{r - l} & 0 & -\frac{r + l}{r - l} & 0 \\
        0 & \frac{2 z_{near}}{t - b} & -\frac{t + b}{t - b} & 0 \\
        0 & 0 & \frac{z_{far}}{z_{far} - z_{near}} & \frac{- z_{far} z_{near}}{z_{far} - z_{near}} & 0 \\
        0 & 0 & 1 & 0
      \end{bmatrix} \\
    \end{align*}

    DirectX, Metal (Symmetrical): when $ \left| l \right| = \left| r \right| $,
    $ \left| t \right| = \left| b \right| $, $ w = r - l, h = t - b $

    \begin{align*}
      \begin{bmatrix}
        \frac{1}{w} & 0 & 0 & 0 \\
        0 & \frac{1}{h} & 0 & 0 \\
        0 & 0 & \frac{z_{far}}{z_{far} - z_{near}} & \frac{- z_{far} z_{near}}{z_{far} - z_{near}} & 0 \\
        0 & 0 & 1 & 0
      \end{bmatrix}
    \end{align*}

    OpenGL (Asymmetrical)

    \begin{align*}
      \begin{bmatrix}
        \frac{2 z_{near}}{r - l} & 0 & \frac{r + l}{r - l} & 0 \\
        0 & \frac{2 z_{near}}{t - b} & \frac{t + b}{t - b} & 0 \\
        0 & 0 & - \frac{z_{far} + z_{near}}{z_{far} - z_{near}} & \frac{- 2 z_{far} z_{near}}{z_{far} - z_{near}} & 0 \\
        0 & 0 & -1 & 0
      \end{bmatrix}
    \end{align*}

    OpenGL (Symmetrical)

    \begin{align*}
      \begin{bmatrix}
        \frac{1}{w} & 0 & 0 & 0 \\
        0 & \frac{1}{h} & 0 & 0 \\
        0 & 0 & - \frac{z_{far} + z_{near}}{z_{far} - z_{near}} & \frac{- 2 z_{far} z_{near}}{z_{far} - z_{near}} & 0 \\
        0 & 0 & -1 & 0
      \end{bmatrix}
    \end{align*}

    \begin{itemize}
      \item $ l $: the left edge of clip space; \textbf{smallest world x value
      to be included into the clip space}
      \item $ r $: right edge of clip space; \textbf{biggest world x value
      to be included into the clip space}
      \item $ t $: top edge of clip space; \textbf{smallest world y value
      to be included into the clip space}
      \item $ b $: bottom edge of clip space; \textbf{biggest world y value
      to be included into the clip space}
      \item $ z_{near} $ the distance from the screen to the eye; \textbf{
      smallest world z value to be included into the clip space}
      \begin{itemize}
        \item Increasing $ z_{near} $; would reduce the size of the image; changing
        $ z_{near} $ distort the image
      \end{itemize}
    \end{itemize}

  \subsubsection{Deriving Matrix}

    \begin{figure}[H]
      \centering
      \begin{tikzpicture}
        \filldraw (0, 0) circle (1pt) node[label={[yshift=-1cm]Eye}] (eye) {};
        \filldraw (4, 0) circle (1pt) node[label={[yshift=-1cm]$ z_{p} $}] (z_p) {};
        \filldraw (8, 0) circle (1pt) node[label={[yshift=-1cm]$ z_{world} $}] (z_world) {};
        \filldraw (8, 4) circle (1pt) node[label={$ y_{world} $}] (y_world) {};
        \filldraw (4, 2) circle (1pt) node[label={[yshift=-1cm]$ y_{p} $}] (y_p) {};

        \draw (eye) -- (z_p) -- (z_world);
        \draw (z_world) -- (y_world) -- (y_p) -- (eye);
      \end{tikzpicture}
    \end{figure}

    \begin{align}
      \frac{y_{p}}{y_{world}} &= \frac{z_{p}}{z_{world}} \\
      y_{clip} &= \frac{z_{p} y_{world}}{z_{world}}
    \end{align}

    This equation also applies to $ x $

    \begin{itemize}
      \item $ z_{p} $ is the plane where everything is projected onto.
      \begin{itemize}
        \item $ z_{p} = -1 $ in OpenGL, WebGL
        \item $ z_{p} = 1 $ in Metal, DirectX
      \end{itemize}
      \item Not invertible
      \item Not an affine transformation (does not preserve the ratio of
      distances)
    \end{itemize}

    The steps below assume \acrshort{ndc} like the ones in Metal, DirectX

    \begin{align*}
      &\begin{bmatrix}
        1 & 0 & 0 & 0 \\
        0 & 1 & 0 & 0 \\
        0 & 0 & 1 & 0 \\
        0 & 0 & 1 & 0
      \end{bmatrix}
      \times
      \begin{bmatrix}
        x \\
        y \\
        z \\
        w
      \end{bmatrix}
      =
      \begin{bmatrix}
        x \\
        y \\
        z \\
        z
      \end{bmatrix}
    \end{align*}

    When graphics APIs convert from \gls{clip-space} to \acrshort{ndc} by
    dividing every component by $ w $ (now $ z $), we have

    \begin{align*}
      \begin{bmatrix}
        x / z \\
        y / z \\
        z / z \\
        z / z
      \end{bmatrix}
      =
      \begin{bmatrix}
        x / z \\
        y / z \\
        1 \\
        1
      \end{bmatrix}
    \end{align*}

    At this stage, the overall shape of objects will appear projected, but
    all the $ z $ are now $ 1 $. As a result, we can no longer perform depth
    testing.

    To solve this, we want to map all $ z_{world} $ in the range
    $ \left[z_{near}, z_{far} \right] $ into $ z_{ndc} $ in
    the range $ \left[0, 1\right] $.

    \begin{align*}
      \begin{bmatrix}
        x_{clip} \\
        y_{clip} \\
        z_{clip} \\
        w_{clip}
      \end{bmatrix}
      &=
      \begin{bmatrix}
        1 & 0 & 0 & 0 \\
        0 & 1 & 0 & 0 \\
        0 & 0 & A & B \\
        0 & 0 & 1 & 0
      \end{bmatrix}
      \times
      \begin{bmatrix}
        x_{world} \\
        y_{world} \\
        z_{world} \\
        w_{world}
      \end{bmatrix} \\
      z_{ndc} &= \frac{z_{clip}}{w_{clip}} = \frac{A z_{world} + B w_{world}}{z_{world}}
    \end{align*}

    We know that $ w_{world} = 1 $

    \begin{align*}
      z_{ndc} &= \frac{z_{clip}}{w_{clip}} = \frac{A z_{world} + B}{z_{world}}
    \end{align*}

    Using the relationships

    \begin{enumerate}
      \item When $ z_{world} = z_{near} $, $ z_{ndc} = 0 $
      \item When $ z_{world} = z_{far} $, $ z_{ndc} = 1 $
    \end{enumerate}

    \begin{align*}
      \begin{cases}
        \frac{A z_{near} + B}{z_{near}} = 0 \\
        \frac{A z_{far} + B}{z_{far}} = 1
      \end{cases}
      &\to
      \begin{cases}
        A z_{near} + B = 0 \\
        A z_{far} + B = z_{far}
      \end{cases} \\
      A &= \frac{z_{far}}{z_{far} - z_{near}} \\
      B &= \frac{- z_{far} z_{near}}{z_{far} - z_{near}}
    \end{align*}

    We must also twist $ x, y $ to compensate for the scaling that results
    from when \acrshort{ndc} is scaled to viewport size.

    The following can be used across graphics APIs, as their \acrshort{ndc}
    are only different in $ z $ axis. The steps of $ x, y $ are the same.

    $ x, y $ transforms goes through two steps of transformations.
    They are first transformed into $ x_{p}, y_{p} $, which are points they
    will be projected on the $ z_{near} $ plane.

    \begin{align*}
      x_{p} &= \frac{z_{near}}{z_{world}} x_{world} & y_{p} &= \frac{z_{near}}{z_{world}} y_{world}
    \end{align*}

    Then these points are transformed into NDC. When $ x_{p} $ is $ r $,
    $ x_{ndc} = 1 $

    \begin{align*}
      x_{ndc} &= \frac{1 - \left( -1 \right)}{r - l} x_{p} + \beta \\
      1 &= \frac{2r}{r - l} + \beta \\
      \beta &= - \frac{r + l}{r - l} \\
      x_{ndc} &= \frac{2}{r - l} x_{p} - \frac{r + l}{r - l}
    \end{align*}

    Substituting $ x_{p} $ gives us

    \begin{align*}
      x_{ndc}
        &= \frac{2}{r - l} \frac{z_{near}}{z_{world}} x_{world} - \frac{r + l}{r - l} \\
        &= \frac{\frac{2 z_{near}}{r - l} x_{world}}{z_{world}} - \frac{r + l}{r - l} \\
        &= \frac{\frac{2 z_{near}}{r - l} x_{world}}{z_{world}} - \frac{\frac{r + l}{r - l} z_{world}}{z_{world}} \\
        &= \frac{\frac{2 z_{near}}{r - l} x_{world} - \frac{r + l}{r - l} z_{world} }{z_{world}}
    \end{align*}

    Putting this into matrix, we have

    \begin{align*}
      \begin{bmatrix}
        \frac{2 z_{near}}{r - l} & 0 & -\frac{r + l}{r - l} & 0 \\
        0 & \frac{2 z_{near}}{t - b} & -\frac{t + b}{t - b} & 0 \\
        ...
      \end{bmatrix}
    \end{align*}

    \href{http://www.songho.ca/opengl/gl_projectionmatrix.html}{Original Answer}
