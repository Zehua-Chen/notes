\chapter{Pipelines}\label{chapter: pipelines}

\begin{enumerate}
  \item \textbf{Vertex}: The majority of transformation pipeline happens here,
  \textbf{except W2V}
  \begin{itemize}
    \item Programmable through shader
    \item Produces vertices in \textbf{\gls{clip-space}}
  \end{itemize}

  \item \textbf{Rasterisation}: produces a set of fragments (which are
  potential pixels); fragments have
  \begin{itemize}
    \item Location in frame buffer (screen location)
    \item Color
    \item Depth
  \end{itemize}

  \item \textbf{Fragment}: the fragment shader would finalize the color
  produced by the rasterisation stage
  \begin{itemize}
    \item Not pixels: each fragment has a 2D location in a raster and a color;
    final value is found by applying hidden surface removal and possibly
    compositing to a set of fragments
    \item Programmable through shader
  \end{itemize}
\end{enumerate}

\section{Clip Space and NDC (Normalized Device Coordinate)}

  \begin{itemize}
    \item \textbf{\Gls{clip-space}}: \glsdesc{clip-space}
    \item \textbf{\Gls{normalized-device-coordinate}, \acrshort{ndc}}:
    \glsdesc{normalized-device-coordinate}
    \begin{itemize}
      \item Independent of view port
    \end{itemize}
  \end{itemize}

\section{Transformation Pipeline}

  \begin{equation}
    \begin{bmatrix}
      x_{s} \\
      y_{s} \\
      0 \\
      1
    \end{bmatrix}
    =
    \begin{bmatrix}
      \text{W2V}
    \end{bmatrix}
    \begin{bmatrix}
      \text{Clipping}
    \end{bmatrix}
    \begin{bmatrix}
      \text{Projection}
    \end{bmatrix}
    \begin{bmatrix}
      \text{View}
    \end{bmatrix}
    \begin{bmatrix}
      \text{Model}
    \end{bmatrix}
    \begin{bmatrix}
      x_{m} \\
      y_{m} \\
      z_{m} \\
      1
    \end{bmatrix}
  \end{equation}

  Transforming model coordinate system to viewport coordinate system
  (\textbf{note that transformations occur from right to left})

  \begin{itemize}
    \item Mostly done on GPU (except \textbf{W2V})
    \item GPU parts are done through the vertex shader
    \item Can also handle colors
  \end{itemize}

  \paragraph{Stages}
  \begin{itemize}
    \item The \textbf{model matrix} map local object coordinates to
    world coordinates (ex. rotation)
    \item Given a camera looking down the z-axis from the origin,
    the \textbf{view matrix} map world coordinates to camera coordinates,
    where camera sits at the origin
    \begin{itemize}
      \item Gives us a view volume that is not a cube
    \end{itemize}

    \item The \textbf{projection stage} map camera coordinates to
    \gls{normalized-device-coordinate}
    \begin{itemize}
      \item The \textbf{projection matrix} map camera coordinates into
      \gls{clip-space} coordinates
      \begin{itemize}
        \item $ x, y $ coordinates of vertices will be twisted to compensate for
        the scaling that occur when \acrshort{ndc} are scaled to window size
      \end{itemize}

      \item Coordinates in \gls{clip-space} ($ x, y, z, w $) are divided by
      $ w $ to give coordinates in \gls{normalized-device-coordinate}
      ($ x, y, z, 1 $)
      \item Graphics API typically performs orthographic projection out of
      the box, but \textbf{depth information is preserved} for hidden surface
      removal
    \end{itemize}

    \item \textbf{Clipping}: only contents within the view volume needs to be
    drawn
    \item The \textbf{W2V (window to view) matrix} map canonical view volume
    to screen space; steps
    \begin{enumerate}
      \item \textbf{Clipping}: only draw stuff in clip space
      \item \textbf{Homogeneous divide}
      \item \textbf{Window to view port}: transform from clip coordinates
      (aka. window coordinates $ -1 \to 1 $) to viewport space
    \end{enumerate}
  \end{itemize}

  \textbf{Every transformations except W2V transformation happen in vertex
  processing stage}

  \subsection{In Unity}

    \begin{itemize}
      \item Transform component on a camera is the view matrix
      \item Camera settings are used to create
      \begin{itemize}
        \item Projection matrix
        \item Viewport matrix
      \end{itemize}
    \end{itemize}

\section{Stage - View}

  View transformations translate and rotate all objects in the scene in such
  a way that \textbf{the view of the camera would appear offset and rotated but
  in face at origin (we can't rotate the viewport)}

  \paragraph{Input}
  \begin{itemize}
    \item $ \vec{eye} $: eye position
    \item $ \vec{at} $: look at position
    \item $ \vec{up} $: up vector
  \end{itemize}

  \paragraph{Output}
  \subparagraph{Deriving X, Y, Z Axis}
  \begin{align}
    \vec{v} &= \vec{z} = \vec{at} - \vec{eye} \\
    \vec{n} &= \vec{x} = \vec{z} \times \vec{up} \\
    \vec{u} &= \vec{y} = \vec{x} \times \vec{z}
  \end{align}

  \subparagraph{Deriving Transformation Matrix}
  $ M $ would place the camera on its intended transform. However, since
  we can't transform the camera, we must move everything in the scene instead
  with $ M^{-1} $.

  Left-handed:

  \begin{align}
    M &= T R \\
    M^{-1} &= \left( T R \right)^{-1} \\
    &= R^{-1} T^{-1} \\
    &=
    \begin{bmatrix}
      \vec{n}_{x} & \vec{u}_{x} & \vec{v}_{x} & 0 \\
      \vec{n}_{y} & \vec{u}_{y} & \vec{v}_{y} & 0 \\
      \vec{n}_{z} & \vec{u}_{z} & \vec{v}_{z} & 0 \\
      0 & 0 & 0 & 1
    \end{bmatrix}^{-1}
    \begin{bmatrix}
      1 & 0 & 0 & \vec{eye}_{x} \\
      0 & 1 & 0 & \vec{eye}_{y} \\
      0 & 0 & 1 & \vec{eye}_{z} \\
      0 & 0 & 0 & 1
    \end{bmatrix}^{-1} \\
    &=
    \begin{bmatrix}
      \vec{n}_{x} & \vec{n}_{y} & \vec{n}_{z} & 0 \\
      \vec{u}_{x} & \vec{u}_{y} & \vec{u}_{z} & 0 \\
      \vec{v}_{x} & \vec{v}_{y} & \vec{v}_{z} & 0 \\
      0 & 0 & 0 & 1
    \end{bmatrix}
    \begin{bmatrix}
      1 & 0 & 0 & -\vec{eye}_{x} \\
      0 & 1 & 0 & -\vec{eye}_{y} \\
      0 & 0 & 1 & -\vec{eye}_{z} \\
      0 & 0 & 0 & 1
    \end{bmatrix} \\
    &=
    \begin{bmatrix}
      \vec{n}_{x} & \vec{n}_{y} & \vec{n}_{z} & -\left( \vec{n} \cdot \vec{eye} \right) \\
      \vec{u}_{x} & \vec{u}_{y} & \vec{u}_{z} & -\left( \vec{u} \cdot \vec{eye} \right) \\
      \vec{v}_{x} & \vec{v}_{y} & \vec{v}_{z} & -\left( \vec{v} \cdot \vec{eye} \right) \\
      0 & 0 & 0 & 1
    \end{bmatrix}
  \end{align}

  Right-handed:

  \begin{align}
    M^{-1} =
    \begin{bmatrix}
      \vec{n}_{x} & \vec{n}_{y} & \vec{n}_{z} & \left( \vec{n} \cdot \vec{eye} \right) \\
      \vec{u}_{x} & \vec{u}_{y} & \vec{u}_{z} & \left( \vec{u} \cdot \vec{eye} \right) \\
      \vec{v}_{x} & \vec{v}_{y} & \vec{v}_{z} & \left( \vec{v} \cdot \vec{eye} \right) \\
      0 & 0 & 0 & 1
    \end{bmatrix}
  \end{align}

\subimport{./}{projection}

\section{Stage - Window to Viewport}

  This is the transformation used by systems where the
  $ x \in \left[ -1, 1 \right], y \in \left[ -1, 1 \right] $ are mapped to
  the width and height of the viewport.

  \begin{enumerate}
    \item Translate the lower left corner to $ \left( 0, 0 \right) $
    \item Scale the upper right corner to $ \left( w - 1, h - 1 \right) $
  \end{enumerate}

  \begin{align}
    \text{W2V } &=
    \begin{bmatrix}
      \frac{w - 1}{2} & 0 & 0 \\
      0 & \frac{h - 1}{2} & 0 \\
      0 & 0 & 1
    \end{bmatrix}
    \begin{bmatrix}
      1 & 0 & 1 \\
      0 & 1 & 1 \\
      0 & 0 & 1
    \end{bmatrix}
    \begin{bmatrix}
      x \\
      y \\
      1
    \end{bmatrix} \\
    &=
    \begin{bmatrix}
      \frac{w - 1}{2} & 0 & \frac{w - 1}{2} \\
      0 & \frac{h - 1}{2} & \frac{h - 1}{2} \\
      0 & 0 & 1
    \end{bmatrix}
    \begin{bmatrix}
      x \\
      y \\
      1
    \end{bmatrix}
  \end{align}

  \begin{itemize}
    \item $ w $: viewport width
    \item $ h $: viewport height
  \end{itemize}

  \subsection{Continuous Space}

    Even though pixels are discrete units (you can't have half of a pixel),
    it can be helpful to still consider pixel spaces to be continuous; WebGL
    and DirectX considers pixel centers to be
    $ \left( x + 0.5, y + 0.5 \right) $

\section{Hidden Surface Removal}

  \begin{itemize}
    \item \textbf{Painter's algorithm}: draw farther object first
    \begin{itemize}
      \item Expensive to sort
      \item Depth cycle
    \end{itemize}

    \item \textbf{Z-Buffer}: keep a z buffer on pixels, while drawing,
    only write to pixel if the object being drawn has a z value that is
    closer than the z value of the pixel
    \begin{itemize}
      \item More popular today; not used in the past for memory issues
      \item If two z values are too close, it's hard to determine which is
      closer
    \end{itemize}

    \item \textbf{Aka. depth test}
  \end{itemize}
