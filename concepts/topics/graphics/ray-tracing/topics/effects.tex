\chapter{Effects}

\section{Depth of Field}

  \textbf{\Gls{depth of field}}: \glsdesc{depth of field}

  \begin{itemize}
    \item Aka. defocus blur
    \item Pinhole camera has a infinite \gls{depth of field}
    \item Objects are only in focus at a single distance called
    \textbf{focal distance}
  \end{itemize}

  \subsection{Thin Lens Theory}

    \textbf{\Gls{thin lens theory}}: \glsdesc{thin lens theory}

    \begin{itemize}
      \item \textbf{\Gls{image plane}}: \glsdesc{image plane}
      \item \textbf{\Gls{focal plane}}: \glsdesc{focal plane}
      \item \Gls{image plane} and \gls{focal plane} exist in pairs
      \begin{itemize}
        \item Moving the \gls{image plane} forward will move the
        \gls{focal plane} forward
      \end{itemize}

      \item Points not on the focal plane will generate a
      \textbf{circle of confusion} on the \gls{image plane}
      \begin{itemize}
        \item The further a point $ q $ gets from the \gls{focal plane}, the
        more larger the circle of confusion
      \end{itemize}

      \item Thins lens camera has zero \gls{depth of field}
      \begin{itemize}
        \item We have a \gls{focal plane}, which is a specific depth value
        \item Traditional camera has a finite grain size which allows a range of
        distances to be in focus
        \item Digital camera is in focus when hte \textbf{circle of confusion}
        is smaller than a pixel
      \end{itemize}
    \end{itemize}

  \subsection{Simulating Thin Lens}

    \begin{enumerate}
      \item Add a disc centered at the eye point, parallel to the view plane
      \begin{itemize}
        \item No thickness
      \end{itemize}

      \item Add a focal plane behind the view plane
      \begin{itemize}
        \item The distance betwenen the disc and the \gls{focal plane} is $ f $,
        focal distance
        \item $ d $, distance from eye to view is decoupled
      \end{itemize}
    \end{enumerate}

    \begin{algorithm}
      \caption{Simulating Thin Lens}
      \ForEach{pixel} {
        Shoot a center ray from through the eye point and hit $ p $ on
        focal plane\;
        Choose a random point on lens as $ O $\;
        Shoot primary rays from $ O $ through $ p $\;
        Average samples to get the final colors\;
      }
    \end{algorithm}

    \begin{itemize}
      \item If multiple primary rays ar eshot through a single $ p $, then
      objects not on the focal plane are anti-aliased, but objects on the focal
      plane are not anti-aliased
      \item To handle the latter case, use sampling to create a different
      center rays for each primary rays
      \item Obtaining $ p $ can be optimized using similar triangles
      \begin{align}
        p &= \left( p_{x}, p_{y}, -f \right) \\
        p_{s} &= \left( p_{sx}, p_{sy}, -d \right) \\
        l_{s} &= \left( l_{sx}, l_{sy}, 0 \right) \\
        p &= \left( p_{sx} \frac{f}{d}, p_{sy} \frac{f}{d}, -f \right) \\
        d_{r} &= p - l_{s} \\
        &= \left( p_{x} - l_{sx} \right) u
          - \left( p_{y} - l_{sy} \right) v
          - fw
      \end{align}
      \begin{itemize}
        \item $ p_{s} $ is a point on the view plane
        \item $ f $ focal length
        \item $ d $ view plane distance
        \item $ l_{s} $: randomly generated point on the disk at eye point
        \item $ d_{r} $: direction of primary ray
        \item $ u, v, w $ are vectors used in camera frame transformation
        from chapter \ref{chapter: camera} on page \pageref{chapter: camera}
      \end{itemize}

    \end{itemize}

    \begin{algorithm}
      \caption{Random Position in Unit Disk}
      \ForEach{pixel} {
        \While{true} {
          Generate a random point $ p $ with $ x, y \in \left[ -1, 1 \right] $\;
          \If{$ p $'s length squared is bigger than $ 1 $} {
            continue\;
          }

          \Return{$ p $}
        }
      }
    \end{algorithm}

    \begin{itemize}
      \item The larger the lens radius, the greather amount of space to choose
      from, the more pixels a point in world would contribute to, the blurrier
      the image will be, the more samples will be required for quality image
    \end{itemize}
