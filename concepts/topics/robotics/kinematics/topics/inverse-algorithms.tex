\chapter{Inverse Kinematics Algorithms}

\section{Cyclic Coordinate Descent}

  Solve 1-DOF IK problem is easier. Iteratively cycle through each joint
  one at a time until full convergence.

  \subsection{Revolute Joint}

    \begin{itemize}
      \item $ v_{in} $: vector between the joint and the end effector normalized
      \item $ v_{it} $: vector between target position and end effector and joint
      \item $ o_{t}, o_{n} $ change as the robot moves
      normalized
      \item If $ \sign\left( v_{in} \times v_{it} \right) $ is positive, then
      $ \Delta \theta_{i} $ is positive, and vice versa
    \end{itemize}

  \subsection{Prismatic Joint}

  \subsection{Considerations}

    \begin{itemize}
      \item Joint ordering does not affect convergence
      \item Start near the base: similar to human leg movement (start at hip)
      \begin{itemize}
        \item All links end up moving, links near end effector will move the most
      \end{itemize}

      \item Start near end effector: similar to human hand and arm
      \begin{itemize}
        \item Less disruption to the links closer to the base
      \end{itemize}

      \item Motions not guaranteed to be smooth
      \item Convergence difficult to targets that require singular or near-signular
      configurations
    \end{itemize}

\section{FK Approximation}

\section{Newton's Method}
