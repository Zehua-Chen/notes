\chapter{Velocity Kinematics}

\begin{itemize}
  \item End effector exhibits \emph{both linear and angular velocities}
  \item Sum of velocity contributions frmo individual joints
  \item Rotate induces translation of points on the rigid body
\end{itemize}

\section{Joints}

  \begin{itemize}
    \item \emph{Prismatic joints}
    \begin{itemize}
      \item Translate alongs $ z_{i - 1} $ axis with velocity $ \dot{d_{i}} $
      \item Contributes \emph{zero angular velocity}
    \end{itemize}

    \item \emph{Revolute joints}
    \begin{itemize}
      \item Rotate about $ z_{i - 1} $ with velocity $ \theta $
      \item Translate end effector
    \end{itemize}
  \end{itemize}

\section{Velocity}

  \subsection{Angular}

    \begin{align}
      \vec{\omega} &= \dot{\theta} k \\
      \vec{\omega}^{0}_{n}
        &= \sum_{i = 1}^{n} \vec{\omega}^{0}_{i} \\
        &= \sum_{i = 1}^{n} p_{i} \dot{q_{i}} z^{0}_{i - 1} \\
    \end{align}

    \begin{align}
      J_{\omega} &=
        \begin{bmatrix}
          J_{\omega 1} & ... & J_{\omega n}
        \end{bmatrix}  \\
      \omega
        &=
        \begin{bmatrix}
          \omega_{x} \\
          \omega_{y} \\
          \omega_{z}
        \end{bmatrix} \\
        &=
        \begin{bmatrix}
          p_{1} z^{0}_{0} ... p_{n} z^{0}_{n - 1}
        \end{bmatrix}
        \begin{bmatrix}
          \dot{q_{1}} \\
          \vdots \\
          \dot{q_{n}}
        \end{bmatrix} \\
        &= J_{w} \dot{q}
    \end{align}

    $ k $ is a unit vector, $ J_{w} $ is angular velocity jacobian,
    $ p_{1} z^{0}_{0} $ has a size of $ 3 \times $

    \begin{itemize}
      \item Angular velocity of a rotating body is the vector $ \vec{\omega} $
      \begin{itemize}
        \item \emph{Magnitude} is the rate of rotation
        \item \emph{Direction} is the axis of rotation
      \end{itemize}

      \item $ p_{i} = 1 $ if joint is revolute
      \item $ p_{i} = 0 $ if joint is prismatic
    \end{itemize}

  \subsection{Linear Velocity}

    \begin{align}
      \vec{v}
        &= \omega \times \vec{r} \\
        &= \dot{\theta_{i}} z_{i - 1} \times \left( o_{n} - o_{i - 1} \right)
    \end{align}

    \begin{align}
      J_{v}
        &=
        \begin{bmatrix}
          J_{v1} & ... & J_{vn}
        \end{bmatrix} \\
      v &=
        \begin{bmatrix}
          v_{x} \\
          v_{y} \\
          v_{z}
        \end{bmatrix} \\
        &=
        \begin{bmatrix}
          J_{v1} & ... & J_{vn}
        \end{bmatrix}
        \begin{bmatrix}
          \dot{q_{1}} \\
          \vdots \\
          \dot{q_{n}}
        \end{bmatrix} \\
        &= J_{v} \dot{q}
    \end{align}

    $ o_{n} $ is origin of end effector frame.

    \begin{itemize}
      \item $ \vec{v} $: linear velocity
      \item $ \vec{r} $ is a vector from a frame origin to the point
    \end{itemize}

\section{Jacobian Matrix}

  \begin{align}
    \xi
      &=
      \begin{bmatrix}
        v \\
        \omega
      \end{bmatrix} \\
      &=
      \begin{bmatrix}
        J_{v} \left( q \right) \\
        J_{\omega} \left( q \right)
      \end{bmatrix}
      \dot{q} \\
      &= J\left( q \right) \dot{q}
  \end{align}

  \begin{itemize}
    \item Jacobian matrix is a linear mapping between joint velocities
    $ \dot{q} $ and the end effector velocities $ \xi = (v, w)^{T} $
    \item \emph{Configuration dependent}, i.e. a function of $ q $
    \item Prismatic joint: $ J_{v, i} = z_{i - 1}, J_{\omega, i} = 0 $
    \item Revolute joint:
    $ J_{v, i} = z_{i - 1} \times \left( o_{n} - o_{i - 1} \right), J_{\omega, i} = z_{i - 1} $
    \item Generally
    \begin{itemize}
      \item $ J_{v} $ is $ 3 \times n $
      \item $ J_{w} $ is $ 3 \times n $
      \item $ J $ is $ 6 \times n $
    \end{itemize}

    \item To compute these quantities, we require the transformation from
    forward kinematics
    \begin{itemize}
      \item Derive linear velocities by directly differentiating the FK
    \end{itemize}
  \end{itemize}

  \subsection{Rank and Singularities}

    \begin{itemize}
      \item Each column of $ J $ indicates the instantaneous direction of end
      effector movement due to unit velocity at the corresponding joint (and 0 at the others)
      \item Column rank of $ J $ = number of independent degrees of freedom $ \le min(6, n ) $
      \item Configurations at which $ J, J_{v}, J_{\omega} $ lose rank are singular configurations
    \end{itemize}

  \subsection{Subspace}

    \begin{itemize}
      \item Column space is the set of all possible end effector velocities
      \item Null space is hte set of all joint velocities producing 0 end effector movement
      \begin{itemize}
        \item Only nontrivial when $ J $ is not full column rank (linearly dependent columns)
      \end{itemize}
    \end{itemize}
